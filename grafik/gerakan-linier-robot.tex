
\begin{figure}[ht]
  \centering
  \begin{tikzpicture}
    \begin{axis}[
        height=0.35\textwidth,
        width=0.9\textwidth,
        ylabel=Jarak (meter),
        xlabel=Percobaan Ke-,
        legend style={
          at={(0.5,1.5)},
          anchor=north,
          legend columns=-1,
        },
        ymajorgrids,
        bar width=4pt,
        ybar=0pt,
        xmin=0.1,
        xmax=12.9,
        ymin=0,
        xtick distance=1,
        ytick distance=1,
      ]
      \addplot table[x=index,y=expecteddistance,col sep=comma]{data/gerakan_linier_robot.csv};
      \addplot table[x=index,y=measureddistance,col sep=comma]{data/gerakan_linier_robot.csv};
      \addplot table[x=index,y=odometrydistance,col sep=comma]{data/gerakan_linier_robot.csv};
      \legend{Expected,Measured,Odometry}
    \end{axis}
  \end{tikzpicture}
  \caption{Grafik estimasi jarak posisi akhir dari gerakan linier pada \emph{real robot}.}
  \label{fig:grafikgerakanlinierrobot}
\end{figure}
