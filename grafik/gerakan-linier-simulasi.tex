
\begin{figure} [ht]
  \centering
  \begin{tikzpicture}
    \begin{axis}[
        height=0.4\textwidth,
        width=0.8\textwidth,
        ylabel=X (meter),
        xticklabels={,,},
        ymajorgrids,
        bar width=3pt,
        ybar=0pt,
        xmin=-1,
        xmax=13,
      ]
      \addplot table[x=index,y=expectedx,col sep=comma]{data/gerakan_linier_simulasi.csv};
      \addplot table[x=index,y=measuredx,col sep=comma]{data/gerakan_linier_simulasi.csv};
      \addplot table[x=index,y=odometryx,col sep=comma]{data/gerakan_linier_simulasi.csv};
    \end{axis}
  \end{tikzpicture}
  \begin{tikzpicture}
    \begin{axis}[
        height=0.4\textwidth,
        width=0.8\textwidth,
        ylabel=Y (meter),
        xticklabels={,,},
        legend style={
          at={(0.5,-0.2)},
          anchor=north,
          legend columns=-1,
        },
        ymajorgrids,
        bar width=3pt,
        ybar=0pt,
        xmin=-1,
        xmax=13,
      ]
      \addplot table[x=index,y=expectedy,col sep=comma]{data/gerakan_linier_simulasi.csv};
      \addplot table[x=index,y=measuredy,col sep=comma]{data/gerakan_linier_simulasi.csv};
      \addplot table[x=index,y=odometryy,col sep=comma]{data/gerakan_linier_simulasi.csv};
      \legend{Expected,Measured,Odometry}
    \end{axis}
  \end{tikzpicture}
  \caption{Grafik estimasi posisi dari gerakan linier di simulasi.}
  \label{fig:grafikgerakanlinearsimulasi}
\end{figure}
