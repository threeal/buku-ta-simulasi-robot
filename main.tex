\title{Buku Tugas Akhir ITS}
\author{Maulana, Alfi}

\documentclass[10pt,twoside]{report}
\usepackage[a5paper,top=25mm,left=25mm,right=20mm,bottom=25mm]{geometry}
\usepackage[singlespacing]{setspace}
\usepackage[indonesian]{babel}
\usepackage[pdfauthor={\@author},bookmarksnumbered,pdfborder={0 0 0}]{hyperref}
\usepackage[utf8]{inputenc}
\usepackage[table,xcdraw]{xcolor}
\usepackage[numbers]{natbib}
\usepackage{changepage}
\usepackage{enumitem}
\usepackage{eso-pic}
\usepackage{etoolbox}
\usepackage{graphicx}
\usepackage{lipsum}
\usepackage{lmodern}
\usepackage{longtable}
\usepackage{multido}
\usepackage{tabularx}
\usepackage{wrapfig}

\patchcmd{\cleardoublepage}{\hbox{}}{
  \thispagestyle{empty}
  \vspace*{\fill}
  \begin{center}\textit{[Halaman ini sengaja dikosongkan]}\end{center}
  \vfill}{}{}

\usepackage{fancyhdr}
\fancyhf{}
\renewcommand{\headrulewidth}{0pt}
\pagestyle{fancy}
\fancyfoot[CE,CO]{\thepage}
\patchcmd{\chapter}{plain}{fancy}{}{}
\patchcmd{\chapter}{empty}{plain}{}{}

\usepackage{titlesec}
\titleformat{\chapter}[display]{\bfseries\Large}{BAB \centering\Roman{chapter}}{0ex}{\vspace{0ex}\centering}
\titleformat{\section}{\bfseries\large}{\MakeUppercase{\thesection}}{1ex}{\vspace{1ex}}
\titleformat{\subsection}{\bfseries\large}{\MakeUppercase{\thesubsection}}{1ex}{}
\titleformat{\subsubsection}{\bfseries\large}{\MakeUppercase{\thesubsubsection}}{1ex}{}
\titlespacing{\chapter}{0ex}{0ex}{4ex}
\titlespacing{\section}{0ex}{1ex}{0ex}
\titlespacing{\subsection}{0ex}{0.5ex}{0ex}
\titlespacing{\subsubsection}{0ex}{0.5ex}{0ex}

\usepackage{listings}
\definecolor{comment}{RGB}{0,128,0}
\definecolor{string}{RGB}{255,0,0}
\definecolor{keyword}{RGB}{0,0,255}
\lstdefinestyle{codestyle}{
  commentstyle=\color{comment},
  stringstyle=\color{string},
  keywordstyle=\color{keyword},
  basicstyle=\footnotesize\ttfamily,
  numbers=left,
  numberstyle=\tiny,
  numbersep=5pt,
  frame=lines,
  breaklines=true,
  prebreak=\raisebox{0ex}[0ex][0ex]{\ensuremath{\hookleftarrow}},
  showstringspaces=false,
  upquote=true,
  tabsize=2,
}
\lstset{style=codestyle}

\begin{document}

  \newcommand\covercontents{sampul/konten-id.tex}

  \AddToShipoutPictureBG*{
  \AtPageLowerLeft{
    \hspace{-3.5mm}
    \raisebox{0mm}{
      \includegraphics[width=\paperwidth,height=\paperheight]{sampul/gambar/sampul-luar.png}
    }
  }
}

\thispagestyle{empty}

\newgeometry{
  top=95mm,
  left=25mm,
  right=20mm,
  bottom=25mm
}

\begin{flushleft}

  \sffamily
  \color{white}
  \fontseries{bx}
  \selectfont

  \input{\covercontents}

\end{flushleft}

\restoregeometry


  \setcounter{page}{1}

  \AddToShipoutPictureBG*{
  \AtPageLowerLeft{
    \hspace{-3.5mm}
    \raisebox{0mm}{
      \includegraphics[width=\paperwidth,height=\paperheight]{sampul/gambar/sampul-dalam.png}
    }
  }
}

\thispagestyle{empty}

\newgeometry{
  top=95mm,
  left=25mm,
  right=20mm,
  bottom=25mm
}

\begin{flushleft}

  \sffamily
  \fontseries{bx}
  \selectfont

  \subimport{.}{\covercontents}

\end{flushleft}

\restoregeometry

  \cleardoublepage

  \renewcommand\covercontents{sampul/konten-en.tex}

  \AddToShipoutPictureBG*{
  \AtPageLowerLeft{
    \hspace{-3.5mm}
    \raisebox{0mm}{
      \includegraphics[width=\paperwidth,height=\paperheight]{sampul/gambar/sampul-dalam.png}
    }
  }
}

\thispagestyle{empty}

\newgeometry{
  top=95mm,
  left=25mm,
  right=20mm,
  bottom=25mm
}

\begin{flushleft}

  \sffamily
  \fontseries{bx}
  \selectfont

  \subimport{.}{\covercontents}

\end{flushleft}

\restoregeometry

  \cleardoublepage

  \setlength{\parindent}{2em}
  \setlength{\parskip}{1ex}

  \begin{center}
  \large
  \textbf{PERNYATAAN KEASLIAN\\TUGAS AKHIR}
\end{center}

\thispagestyle{empty}
\vspace{2ex}

Dengan ini saya menyatakan bahwa isi buku Tugas Akhir dengan judul
``\textbf{Pengembangan Lingkungan Simulasi untuk Pengujian \emph{Socially Assistive Robots} Menggunakan ROS 2 dan Gazebo}''
adalah benar hasil karya intelektual mandiri, diselesaikan tanpa menggunakan bahan-bahan yang tidak diijinkan dan bukan merupakan karya pihak lain yang saya akui sebagai karya sendiri.

Semua referensi yang dikutip maupun dirujuk telah ditulis secara lengkap pada daftar pustaka.
Apabila ternyata pernyataan ini tidak benar, saya bersedia menerima sanksi sesuai peraturan yang berlaku.

\vspace{4ex}

\begin{flushright}
  \begin{tabular}[b]{c}
    Surabaya, Juli 2021\\
    \\
    \\
    \\
    \\
    Muhammad Alfi Maulana Fikri\\
    0721 17 4000 0009
  \end{tabular}
\end{flushright}

  \cleardoublepage

  \begin{center}
	\large
  \textbf{LEMBAR PENGESAHAN}
\end{center}

\thispagestyle{empty}

\begin{center}
  \small

  \textbf{PENGEMBANGAN LINGKUNGAN SIMULASI UNTUK PENGUJIAN \emph{SOCIALLY ASSISTIVE ROBOTS} MENGGUNAKAN ROS 2 DAN GAZEBO}
\end{center}

\begingroup
  \small

  \begin{center}
    Tugas Akhir ini disusun untuk memenuhi salah satu syarat memperoleh gelar Sarjana Teknik di Institut Teknologi Sepuluh Nopember Surabaya
  \end{center}

  \begin{center}
    Oleh: Muhammad Alfi Maulana Fikri (NRP. 0721 17 4000 0009)
  \end{center}

  \begingroup
    \setlength{\tabcolsep}{0pt}
    \noindent
    \begin{tabularx}{\textwidth}{X r}
    Tanggal Ujian : 23 Juli 2021 & Periode Wisuda : September 2021
    \end{tabularx}
  \endgroup

  \begin{center}
    Disetujui Oleh:
  \end{center}

  \begingroup
    \setlength{\tabcolsep}{0pt}
    \noindent
    \begin{tabularx}{\textwidth}{X c}
      Prof. Dr. Ir. Mauridhi Hery Purnomo, M.Eng. & (Pembimbing I) \\
      NIP. 19580916 198601 1 001                  & \\
      & \\
      & \\
      Dr. I Ketut Eddy Purnama, S.T., M.T.        & (Pembimbing II) \\
      NIP. 19690730 199512 1 001                  & \\
      & \\
      & \\
      Mochamad Hariadi, ST., M.Sc., Ph.D.         & (Penguji I) \\
      NIP. 19691209 199703 1 002                  & \\
      & \\
      & \\
      Dr. Supeno Mardi Susiki Nugroho, S.T., M.T. & (Penguji II) \\
      NIP. 19700313 199512 1 001                  & \\
    \end{tabularx}
  \endgroup

  \vspace{4ex}

  \begin{center}
    Mengetahui, \\
    Kepala Departemen Teknik Komputer \\

    \vspace{8ex}

    \underline{Dr. Supeno Mardi Susiki Nugroho, S.T., M.T.} \\
    NIP. 19700313 199512 1 001
  \end{center}
\endgroup

  \cleardoublepage

  \pagenumbering{roman}

  \begin{center}
  \large\textbf{ABSTRAK}
\end{center}

\vspace{2ex}

\begingroup
  \setlength{\tabcolsep}{0pt}
  \noindent
  \begin{tabularx}{\textwidth}{l >{\centering}m{2em} X}
    Nama        &:& Muhammad Alfi Maulana Fikri \\
    Judul       &:&	Pengembangan Lingkungan Simulasi untuk Pengujian \emph{Socially Assistive Robots} Menggunakan ROS 2 dan Gazebo \\
    Pembimbing  &:& 1. Prof. Dr. Ir. Mauridhi Hery Purnomo, M.Eng. \\
                & & 2. Dr. I Ketut Eddy Purnama, S.T., M.T. \\
  \end{tabularx}
\endgroup

Selama beberapa tahun terakhir,
  robot telah mengalami perkembangan yang cukup signifikan.
Salah satu bentuk perkembangan tersebut adalah \emph{socially assistive robots} (SARs) yang mampu memberikan bantuan kepada pengguna dalam bentuk interaksi sosial.
Namun, karena sifatnya yang melibatkan interaksi langsung dengan pengguna,
  pengujian pada SARs akan menjadi sulit dan beresiko.
Untuk itu, pada penelitian ini kami mengajukan lingkungan simulasi untuk pengujian SARs yang dibuat menggunakan simulator Gazebo.
% Di dalam lingkungan simulasi ini,
%   model robot akan diujikan dengan model pengguna serta model-model objek lain secara virtual.
Agar pengujian yang dilakukan di simulasi bisa diterapkan pada robot fisik,
  sistem kontroler yang ada pada robot akan dibuat secara terabstraksi dengan memisah setiap komponen menjadi \emph{nodes} menggunakan ROS 2.
% Dengan adanya abstraksi tersebut,
%   program utama robot dapat digunakan pada berbagai sistem yang ada,
%   terlepas dari sistem itu ada di simulasi maupun ada pada robot fisik.
Hasilnya,
  lingkungan simulasi yang dibuat dapat digunakan untuk melakukan pengujian SARs secara virtual.
selain itu, ketika diujikan pada robot fisik,
  tindakan yang dihasilkan memiliki kesamaan dengan yang dihasilkan oleh model robot di simulasi.

Kata Kunci: Simulasi, \emph{Assistive Robotics}, ROS2, Gazebo.

  \cleardoublepage

  \begin{center}
  \large\textbf{ABSTRACT}
\end{center}

\vspace{2ex}

\begingroup
  \setlength{\tabcolsep}{0pt}
  \noindent
  \begin{tabularx}{\textwidth}{l >{\centering}m{2em} X}
    \emph{Name}     &:& Muhammad Alfi Maulana Fikri \\
    \emph{Title}    &:&	\emph{Development of Simulation Environment for Socially Assistive Robots Testing Using ROS 2 and Gazebo} \\
    \emph{Advisors} &:& 1. Prof. Dr. Ir. Mauridhi Hery Purnomo, M.Eng. \\
                    & & 2. Dr. I Ketut Eddy Purnama, S.T., M.T. \\
  \end{tabularx}
\endgroup

\emph{
  In this study, we propose a simulation environment developed using ROS 2 and Gazebo for socially assistive robots (SARs) testing.
  In this simulation environment, the robot model used will be tested with the user model and other virtual object models.
  To make it easier to transfer the program from simulation to physical robot, the robot controller will be developed separately from the simulation environment in which when testing, both will be connected to each other using ROS 2 interprocess communication system.
  It is expected that the simulation environment created can assist the testing of SARs by minimizing risk, reducing costs, and saving time when compared to conducting direct testing using physical robots.
}

\emph{Keywords}: \emph{Simulation}, \emph{Assistive Robotics}, ROS 2, Gazebo.

  \cleardoublepage

  \begin{center}
  \Large
  \textbf{KATA PENGANTAR}
\end{center}

\vspace{2ex}

Puji dan syukur kehadirat Allah SWT atas segala limpahan berkah, rahmat, serta hidayah-Nya, penulis  dapat menyelesaikan penelitian ini dengan judul
``\textbf{Pengembangan Lingkungan Simulasi untuk Pengujian \emph{Socially Assistive Robots} Menggunakan ROS 2 dan Gazebo}''.
Penelitian ini disusun dalam rangka pemenuhan bidang riset di Departemen Teknik Komputer,
  serta digunakan sebagai persyaratan menyelesaikan pendidikan S1.

Dalam penyusunan buku ini,
  penulis mengucapkan terima kasih kepada Keluarga yang telah memberikan dorongan spiritual dan material dalam penyelesaian penelitian ini.
Terutama kepada Ayah atas didikannya kepada penulis selama ini,
  semoga beliau husnul khatimah di sana, aamiin.

Penulis juga mengucapkan terima kasih kepada Bapak Prof. Dr. Ir. Mauridhi Hery Purnomo, M.Eng.,
  Bapak Dr. I Ketut Eddy Purnama, S.T., M.T.,
  dan Bapak Muhtadin ST., MT. atas arahan dan bimbingan selama pengerjaan penelitian tugas akhir ini.
Serta kepada Bapak-ibu dosen pengajar Departemen Teknik Komputer atas pengajaran dan perhatian yang diberikan kepada penulis selama ini.

Dan terakhir,
  terima kasih kepada rekan-rekan ICHIRO ITS, Robotika ITS, dan B201 crew atas pengalamannya kepada penulis.
Serta kepada rekan-rekan seperjuangan Teknik Komputer 2017, E57, dan penghuni rumah anak TK.

Kesempurnaan hanya milik Allah SWT, untuk itu penulis memohon segenap kritik dan saran yang  membangun.
Semoga penelitian ini dapat memberikan manfaat bagi kita semua, aamiin.

\begin{flushright}
  \begin{tabular}[b]{c}
    Surabaya, Juli 2021\\
    \\
    \\
    \\
    \\
    Muhammad Alfi Maulana Fikri
  \end{tabular}
\end{flushright}

  \cleardoublepage

  \renewcommand*\contentsname{DAFTAR ISI}
  \addcontentsline{toc}{chapter}{\contentsname}
  \tableofcontents
  \cleardoublepage

  \renewcommand*\listfigurename{DAFTAR GAMBAR}
  \addcontentsline{toc}{chapter}{\listfigurename}
  \listoffigures
  \cleardoublepage

  \renewcommand*\listtablename{DAFTAR TABEL}
  \addcontentsline{toc}{chapter}{\listtablename}
  \listoftables
  \cleardoublepage

  \pagenumbering{arabic}

  \chapter{PENDAHULUAN}
\label{chap:pendahuluan}

Penelitian ini di latar belakangi oleh berbagai kondisi yang menjadi acuan.
Selain itu juga terdapat beberapa permasalahan yang akan dijawab sebagai luaran dari penelitian.

\section{Latar Belakang}
\label{sec:latarbelakang}

\textcolor{red}{\lipsum[1]}

\textcolor{red}{\lipsum[3][1-16]}

\section{Permasalahan}
\label{sec:permasalahan}

Dari latar belakang yang telah dipaparkan sebelumnya,
  maka permasalahan yang dapat diambil adalah pengujian SARs memiliki resiko terhadap keselamatan pengguna serta dapat memakan biaya dan waktu yang besar.
Untuk itu, perlu adanya lingkungan simulasi yang dapat mensimulasikan ruangan serta objek-objek yang ada di dalamnya sehingga pengujian SARS dapat dilakukan di lingkungan simulasi tersebut,
  sembari memastikan pengujian yang dilakukan di simulasi juga bisa dilakukan dengan hasil yang sama pada robot fisik.

\section{Tujuan}
\label{sec:Tujuan}

Tujuan dari tugas akhir ini adalah sebagai berikut:

\begin{enumerate}[nolistsep]

  \item Membuat \textcolor{red}{\lipsum[1][1-3]}

  \item \textcolor{red}{\lipsum[2][1-2]}

\end{enumerate}

\section{Batasan Masalah}
\label{sec:batasanmasalah}

Untuk memfokuskan permasalahan yang diangkat maka dilakukan pembatasan masalah.
Batasan-batasan masalah tersebut di antaranya adalah:

\begin{enumerate}[nolistsep]

  \item Mempermudah \textcolor{red}{\lipsum[2][1-3]}

  \item \textcolor{red}{\lipsum[3][1-5]}

  \item \textcolor{red}{\lipsum[4][1-4]}

  \item \textcolor{red}{\lipsum[5][1-3]}

\end{enumerate}

\section{Sistematika Penulisan}
\label{sec:sistematikapenulisan}

Laporan penelitian tugas akhir ini tersusun dalam sistematika dan terstruktur sehingga mudah dipahami dan dipelajari oleh pembaca maupun seseorang yang ingin melanjutkan penelitian ini.
Alur sistematika penulisan laporan penelitian ini yaitu:

\begin{enumerate}[nolistsep]

  \item \textbf{BAB I Pendahuluan}

  Bab ini berisi uraian tentang latar belakang permasalahan, penegasan dan alasan pemilihan judul, sistematika laporan, tujuan, dan metodologi penelitian.

  \vspace{2ex}

  \item \textbf{BAB II Tinjauan Pustaka}

  Bab ini berisi tentang uraian secara sistematis teori-teori yang berhubungan dengan permasalahan yang dibahas pada penelitian ini.
  Teori-teori ini digunakan sebagai dasar dalam penelitian, yaitu informasi terkait \textcolor{red}{\lipsum[1][1-2]}

  \vspace{2ex}

  \item \textbf{BAB III Desain dan Implementasi Sistem}

  Bab ini berisi tentang penjelasan-penjelasan terkait eksperimen yang akan dilakukan, \textcolor{red}{\lipsum[1][1-3]}

  \vspace{2ex}

  \item \textbf{BAB IV Pengujian dan Analisa}

  Bab ini menjelaskan tentang hasil serta analisis yang didapatkan dari pengujian yang dilakukan mulai dari hasil pengujian \textcolor{red}{\lipsum[1][1-2]}

  \vspace{2ex}

  \item \textbf{BAB V Penutup}

  Bab ini merupakan penutup yang berisi kesimpulan yang diambil dari penelitian dan pengujian yang telah dilakukan.
  Saran dan kritik yang membangun untuk pengembangan lebih lanjut juga dituliskan pada bab ini.

\end{enumerate}


  \cleardoublepage

  \chapter{TINJAUAN PUSTAKA}
\label{chap:tinjauanpustaka}

Demi mendukung penelitian ini, dibutuhkan beberapa teori penunjang sebagai bahan acuan dan referensi.
Dengan demikian penelitian ini menjadi lebih terarah.

\subsection{\emph{Socially Assistive Robots} (SARs)}
\label{subsec:sociallyassistiverobots}

\emph{Socially assistive robots} (SARs) merupakan jenis robot dalam bidang \emph{socially assistive robotics} yang menggabungkan aspek yang ada pada \emph{assistive robotics} dan \emph{socially interactive robotics}.
SARs memiliki tujuan yang sama dengan robot di bidang \emph{assistive robotics},
  yakni dalam hal memberikan bantuan kepada pengguna secara \emph{assistive}, namun pada SARs,
  bantuan tersebut secara spesifik diberikan melalui interaksi sosial kepada pengguna.
Karena adanya aspek interaksi sosial tersebut, SARs memiliki tujuan yang sama dengan robot di bidang \emph{socially interactive robotics}.

Rich dan Sidner \citep{cit:rich2009} memaparkan,
  SARs mampu memberikan bantuan kepada pengguna dalam berbagai cakupan.
Cakupan tersebut terdiri atas kemampuan SARs untuk memberikan dukungan fungsional dan kognitif kepada pengguna,
  memberikan kesempatan bagi pengguna untuk meningkatkan partisipasi sosial dan kesehatan psikologis,
  menyediakan pemantauan jarak jauh dan berkelanjutan atas status kesehatan pengguna,
  serta memfasilitasi pengguna untuk melakukan perilaku hidup sehat dan pencapaian tujuan yang berhubungan dengan kesehatan.


  \cleardoublepage

  \chapter{DESAIN DAN IMPLEMENTASI SISTEM}
\label{chap:desainimplementasi}

Pada bab ini akan diuraikan desain dan implementasi dari sistem yang telah dibuat.
Desain dan implementasi yang akan diuraikan ini dimulai dari perancangan model dan lingkungan yang akan digunakan di simulasi,
  lalu integrasi ROS 2 pada simulasi sebagai \emph{Gazebo plugins},
  kemudian pengembangan program \emph{behavior} untuk pengujian,
  dan terakhir integrasi sistem yang ada pada robot fisik melalui pengembangan \emph{node controller} sebagai pengganti dari \emph{Gazebo plugins} yang ada di simulasi.

\subimport{3-desain-implementasi}{1-model-robot.tex}
\subimport{3-desain-implementasi}{2-model-pengguna.tex}
\subimport{3-desain-implementasi}{3-lingkungan-simulasi.tex}
\subimport{3-desain-implementasi}{4-integrasi-plugin.tex}
\subimport{3-desain-implementasi}{5-behavior-node.tex}
\subimport{3-desain-implementasi}{6-controller-node.tex}

  \cleardoublepage

  \chapter{PENGUJIAN DAN ANALISIS}
\label{chap:pengujiananalisis}

Pada penelitian ini, dipaparkan hasil pengujian serta analisis dari desain sistem dan implementasi.
Data yang digunakan dalam pengujian data diambil dari \textcolor{red}{\lipsum[1][1-5]}

  \cleardoublepage

  \chapter{PENUTUP}
\label{chap:penutup}

Pada bab ini akan dipaparkan kesimpulan dari hasil pengujian yang akan menjadi jawaban dari permasalahan yang diangkat oleh penelitian ini.
Selain itu akan dipaparkan juga saran mengenai hal yang bisa dilakukan untuk mengembangkan penelitian ini ke arah yang lebih lanjut.

\subimport{5-penutup}{1-kesimpulan.tex}
\subimport{5-penutup}{2-saran.tex}

  \cleardoublepage

  \renewcommand\bibname{DAFTAR PUSTAKA}
  \addcontentsline{toc}{chapter}{\bibname}
  \bibliographystyle{unsrtnat}
  \bibliography{pustaka/pustaka.bib}
  \cleardoublepage

  \input{lainnya/biografi-penulis.tex}
  \cleardoublepage

\end{document}
