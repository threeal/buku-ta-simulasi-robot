\chapter{HASIL DAN PENGUJIAN}
\label{chap:hasilpengujian}

Pada bab ini akan dipaparkan hasil pengujian serta analisis dari desain dan implementasi sistem yang telah dibuat sebelumnya di bab \ref{chap:desainimplementasi}.
Setiap pengujian yang dilakukan pada penelitian ini diujikan menggunakan model robot yang ada di simulasi dan menggunakan prototipe robot yang diuji secara \emph{real}.

Pengujian yang dilakukan menggunakan model robot yang ada di simulasi dilakukan di lingkungan \emph{outdoor} dan di lingkungan \emph{indoor} seperti yang telah dibuat di bagian \ref{sec:lingkungansimulasi}.
Pengujian tersebut dilakukan di simulator Gazebo dengan menggunakan komputer dengan spesifikasi seperti yang dapat dilihat pada tabel \ref{tb:spesifikasikomputersimulator}.
Sedangkan pengujian yang dilakukan menggunakan prototipe robot secara \emph{real} dilakukan di lingkungan laboratorium AJ403 Teknik Komputer ITS serta menggunakan komputer robot dengan spesifikasi seperti yang dapat dilihat pada tabel \ref{tb:spesifikasikomputerrobot}.

\begin{longtable}{|c|c|}
  \caption{Spesifikasi komputer untuk menjalankan simulator.}
  \label{tb:spesifikasikomputersimulator}\\
  \hline
  OS  & Ubuntu 20.04.2 LTS \\
  \hline
  CPU & Intel i3-8100 (4) @ 3.600GH \\
  \hline
  GPU & NVIDIA GeForce GTX 1050 Ti \\
  \hline
  RAM & 7901 MiB \\
  \hline
\end{longtable}

\begin{longtable}{|c|c|c|}
  \caption{Spesifikasi komputer yang ada pada prototipe robot.}
  \label{tb:spesifikasikomputerrobot}\\
  \hline
  OS  & Ubuntu 20.04.2 LTS \\
  \hline
  CPU & Intel i3-10110U (4) @ 4.100 \\
  \hline
  GPU & Intel UHD Graphics \\
  \hline
  RAM & 3648 MiB \\
  \hline
\end{longtable}

\subimport{4-hasil-pengujian}{1-pengujian-gerakan.tex}
\subimport{4-hasil-pengujian}{2-pengujian-citra-kamera.tex}
\subimport{4-hasil-pengujian}{3-pengujian-depth-camera.tex}
% \subimport{4-hasil-pengujian}{4-pengujian-deteksi-pose.tex}
\subimport{4-hasil-pengujian}{5-pengujian-slam.tex}
