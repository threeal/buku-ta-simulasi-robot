\subsection{Pengujian Pengiriman Citra Kamera pada \emph{Real Robot}}
\label{subsec:citrarobot}

Sama seperti pengujian sebelumnya yang ada di bagian \ref{subsec:citrasimulasi},
  pengujian ini juga dilakukan dengan menjalankan \emph{image viewer node} dan \emph{command-line} \lstinline{$ ros2 topic}.
Hanya saja, sebagai ganti dari \emph{node} \lstinline{/camera_plugin} yang ada di simulasi,
  data citra kamera yang dikirim akan berasal dari \emph{V4L2 camera node}.
Seperti yang dapat dilihat pada gambar \ref{fig:rosgraphcamera},
  \emph{node} \lstinline{/v4l2_camera} akan mengirimkan \emph{topic} \lstinline{/image_raw} yang berisi citra kamera,
  setelah itu \emph{node} \lstinline{/image_viewer} akan menerima citra tersebut dan menampilkannya dalam bentuk GUI.

\begin{figure}[ht]
  \centering
  \includegraphics[width=0.8\textwidth,keepaspectratio]{gambar/rosgraph-camera.png}
  \caption{Relasi antar-\emph{node} dari pengujian pengiriman citra kamera pada \emph{real robot}.}
  \label{fig:rosgraphcamera}
\end{figure}

Pengujian ini dilakukan dengan berbagai macam konfigurasi resolusi citra yang dikirim.
Hasil pengujian ini bisa dilihat pada tabel \ref{tb:pengirimancitrasimulasi}.
Sama seperti pengujian yang ada di simulasi, pada tabel tersebut \emph{width} dan \emph{height} merupakan resolusi citra yang dikirimkan,
  \emph{size} merupakan hasil perkalian resolusi dengan jumlah \emph{channel},
  \emph{delay} merupakan selang waktu yang dibutuhkan sebelum citra sampai ke penerima,
  dan \emph{rate} merupakan frekuensi pengiriman citra.

  \begin{longtable}{|c|c|c|c|c|}
  \caption{Hasil \emph{delay} dan frekuensi dari pengiriman citra kamera pada \emph{real robot}.}
  \label{tb:pengirimancitrarobot}
  \\ \hline \rowcolor[HTML]{E0E0E0}
  Width & Height & Size (kb) & Delay (ms) & Rate (hz)
  \csvreader[head to column names]{data/pengiriman_citra_robot.csv}{}{
    \\ \hline
    \width & \height & \size & \delay & \rate
  }
  \\ \hline
\end{longtable}

\begin{longtable}{|c|c|c|c|c|}
  \caption{Hasil \emph{delay} dan frekuensi dari pengiriman citra kamera pada \emph{real robot}.}
  \label{tb:pengirimancitrarobot}
  \\ \hline \rowcolor[HTML]{E0E0E0}
  Width & Height & Size (kb) & Delay (ms) & Rate (hz)
  \csvreader[head to column names]{data/pengiriman_citra_robot.csv}{}{
    \\ \hline
    \width & \height & \size & \delay & \rate
  }
  \\ \hline
\end{longtable}


Dari data yang dihasilkan oleh pengujian ini,
  ketika dibentuk menjadi grafik seperti yang dapat dilihat pada gambar \ref{fig:grafikpengirimancitrasimulasi},
  diketahui bahwa semakin besar ukuran data yang dikirimkan maka \emph{delay} akan cenderung naik dan frekuensi akan cenderung turun,
  sama seperti yang dihasilkan oleh pengujian di simulasi.
Pada pengujian ini, dapat diketahui juga agar \emph{delay} dan frekuensi yang dihasilkan stabil,
  data yang dikirimkan hendaknya ada di bawah 1000 KB (dalam hal ini adalah citra dengan resolusi 640x480),
  karena pada titik tersebut frekuensi yang dikirimkan tidak jauh berbeda dari spesifikasi fps dari kamera yang digunakan \emph{real robot} (Logitech C922).
