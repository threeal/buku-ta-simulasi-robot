\subsection{Pengujian Pengiriman Citra Kamera Antar-perangkat pada \emph{Real Robot}}
\label{subsec:citraantarperangkatrobot}

Pengujian pengiriman citra kamera antar-perangkat pada \emph{real robot} memiliki kesamaan proses dengan yang dilakukan di pengujian sebelumnya yang ada di bagian \ref{subsec:citraantarperangkatsimulasi},
  dimana di pengujian ini, \emph{image viewer node} dan \emph{command line} \lstinline{$ ros2 topic} dijalankan di perangkat berbeda yang terhubung menggunakan jaringan \emph{ethernet} dengan perangkat yang menjalankan \emph{V4L2 camera node}.

Sama seperti pengujian-pengujian sebelumnya,
  pengujian ini juga dilakukan dengan berbagai macam konfigurasi resolusi citra yang dikirim.
Hasil pengujian ini bisa dilihat pada tabel \ref{tb:pengirimancitraantarperangkatrobot}.
Pada tabel tersebut \emph{width} dan \emph{height} merupakan resolusi citra yang dikirimkan,
  \emph{size} merupakan ukuran data yang dikirim,
  \emph{delay} merupakan selang waktu yang dibutuhkan untuk sampai ke perangkat penerima,
  dan \emph{rate} merupakan frekuensi pengiriman citra.

\begin{longtable}{|c|c|c|c|c|c|}
  \caption{Hasil \emph{delay} dan frekuensi dari pengiriman citra kamera antar-perangkat pada \emph{real robot}.}
  \label{tb:pengirimancitraantarperangkatrobot}
  \\ \hline \rowcolor[HTML]{E0E0E0}
  \multicolumn{3}{|c|}{Resolution} &
  \multicolumn{1}{|c|}{Delay} &
  \multicolumn{2}{|c|}{Rate}
  \\ \hline \rowcolor[HTML]{E0E0E0}
  Width & Height & Size (KB) & ms & hz & percent
  \csvreader[head to column names]{data/pengiriman_citra_antarperangkat_robot.csv}{}{
    \\ \hline
    \width & \height & \size & \delay & \rate & \ratepercent
  }
  \\ \hline
\end{longtable}



\begin{figure}[ht]
  \centering
  \begin{tikzpicture}
    \begin{axis}[
        height=0.3\textwidth,
        width=0.9\textwidth,
        ylabel=Delay (ms),
        ymajorgrids,
        ymin=0,
        ymax=125,
      ]
      \addplot table[x=size,y=delay,col sep=comma]{data/pengiriman_citra_antarperangkat_robot.csv};
    \end{axis}
  \end{tikzpicture}
  \begin{tikzpicture}
    \begin{axis}[
        height=0.3\textwidth,
        width=0.9\textwidth,
        xlabel=Data Size (KB),
        ylabel=Rate (hz),
        ymajorgrids,
        ymin=0,
        ymax=40,
      ]
      \addplot table[x=size,y=rate,col sep=comma]{data/pengiriman_citra_antarperangkat_robot.csv};
    \end{axis}
  \end{tikzpicture}
  \caption{Grafik \emph{delay} dan frekuensi dari pengiriman citra antar-perangkat pada \emph{real robot}.}
  \label{fig:grafikpengirimancitraantarperangkatrobot}
\end{figure}


Seperti yang dapat dilihat pada gambar \ref{fig:grafikpengirimancitraantarperangkatrobot},
  dari data yang dihasilkan oleh pengujian ini dapat diketahui bahwa semakin besar ukuran data yang dikirimkan,
  maka \emph{delay} akan cenderung naik dan frekuensi akan cenderung turun.
Pada pengujian ini, dapat diketahui bahwa \emph{delay} yang dihasilkan cukup besar,
  sekitar 100 ms, dan semakin naik dengan bertambahnya ukuran data yang dikirim.
Sedangkan untuk frekuensi,
  nilai yang dapat ditoleransi adalah ketika data yang dikirim berada di bawah 1000 KB,
  diatas itu, frekuensi yang dikirimkan dapat turun drastis hingga 5 hz.
Walaupun begitu, terlepas dari perbedaan \emph{delay} dan frekuensi yang dihasilkan antara pengujian di simulasi dan pada \emph{real robot},
  dapat diketahui bahwa pengiriman data dari kedua sistem tersebut dapat diterima secara abstrak oleh program yang sama (\emph{image viewer node}) di perangkat manapun selama terkoneksi di dalam satu jaringan yang sama.
