\subsection{Pengujian Gerakan Putar dan Estimasi Orientasi di Simulasi}
\label{subsec:putarsimulasi}

Pengujian gerakan putar dan estimasi orientasi di simulasi dilakukan dengan menjalankan \emph{node} yang sama seperti yang ada pada pengujian di bagian \ref{subsec:liniersimulasi}.
Perbedaannya, pengujian ini dilakukan pada beberapa nilai kecepatan putar di sumbu Z selama 10 detik.

Hasil pengujian ini bisa dilihat pada tabel \ref{tb:gerakanputarsimulasi}.
Pada tabel tersebut, nilai yang ada di kolom \emph{speed} adalah besar kecepatan yang diatur pada \emph{topic} \lstinline{/cmd/vel},
  nilai yang ada di kolom \emph{estimated} didapatkan dari perkalian besar kecepatan putar dengan durasi pengujian,
  nilai yang ada di kolom \emph{measured} didapatkan dari orientasi model yang ada di simulasi,
  dan nilai yang ada di kolom \emph{odometry} didapatkan dari data orientasi yang ada pada \emph{topic} \lstinline{/odom}.

\begin{longtable}{|c|c|c|c|c|}
  \caption{Hasil estimasi orientasi dari gerakan putar pada robot di simulasi selama 3 detik.}
  \label{tb:gerakanputarsimulasi}
  \\ \hline \rowcolor[HTML]{E0E0E0}
  Speed &
  Expected &
  Measured &
  \multicolumn{2}{|c|}{Odometry}
  \\ \hline \rowcolor[HTML]{E0E0E0}
  Z (rad/s) &
  Z (deg) &
  Z (deg) &
  Z (deg) & Error
  \csvreader[head to column names]{data/gerakan_putar_simulasi.csv}{}{
    \\ \hline
    \speed &
    \expected &
    \measured &
    \odometry & \odometryerror
  }
  \\ \hline
\end{longtable}


Dari data yang dihasilkan oleh pengujian ini dapat diketahui bahwa gerakan putar yang diperintahkan kepada robot cenderung menghasilkan orientasi robot yang sesuai dengan orientasi perkiraan (\emph{estimated orientation}).
Lebih lanjut, ketika hasil tersebut ditampilkan sebagai grafik,
  seperti yang terlihat pada gambar \ref{fig:grafikgerakanputarsimulasi},
  hasil yang didapatkan cenderung memiliki arah positif negatif yang sesuai dengan yang diharapkan (\emph{expected}).


\begin{figure}[ht]
  \centering
  \begin{tikzpicture}
    \begin{axis}[
        height=0.38\textwidth,
        width=0.9\textwidth,
        ylabel=Z (degree),
        xlabel=Index,
        legend style={
          at={(0.5,-0.5)},
          anchor=north,
          legend columns=-1,
        },
        ymajorgrids,
        bar width=5pt,
        ybar=0pt,
        xmin=0.1,
        xmax=6.9,
        xtick distance=1,
      ]
      \addplot table[x=index,y=expected,col sep=comma]{data/gerakan_putar_simulasi.csv};
      \addplot table[x=index,y=measured,col sep=comma]{data/gerakan_putar_simulasi.csv};
      \addplot table[x=index,y=odometry,col sep=comma]{data/gerakan_putar_simulasi.csv};
      \legend{Expected,Measured,Odometry}
    \end{axis}
  \end{tikzpicture}
  \caption{Grafik estimasi orientasi dari gerakan putar pada robot di simulasi.}
  \label{fig:grafikgerakanputarsimulasi}
\end{figure}

