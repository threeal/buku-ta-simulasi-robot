\subsection{Pengujian Gerakan Putar dan Estimasi Orientasi pada \emph{Real Robot}}
\label{subsec:putarrobot}

Sama seperti pada pengujian simulasi di bagian sebelumnya,
  Pengujian gerakan putar dan estimasi orientasi pada \emph{real robot} juga dilakukan dengan menjalankan \emph{node} yang sama seperti yang ada pada pengujian gerakan linier di bagian \ref{subsec:linierrobot}.
Perbedaannya, pengujian ini dilakukan dengan menggunakan beberapa nilai kecepatan putar di sumbu Z selama 3 detik.

\begin{longtable}{|c|c|c|c|c|}
  \caption{Hasil estimasi orientasi dari gerakan putar pada \emph{real robot} selama 3 detik.}
  \label{tb:gerakanputarrobot}
  \\ \hline \rowcolor[HTML]{E0E0E0}
  \textbf{Speed} &
  \textbf{Expected} &
  \textbf{Measured} &
  \multicolumn{2}{|c|}{\textbf{Odometry}}
  \\ \hline \rowcolor[HTML]{E0E0E0}
  \textbf{Z (rad/s)} &
  \textbf{Z (deg)} &
  \textbf{Z (deg)} &
  \textbf{Z (deg)} & \textbf{Error}
  \csvreader[head to column names]{data/gerakan_putar_robot.csv}{}{
    \\ \hline
    \speed &
    \expected &
    \measured &
    \odometry & \odometryerror
  }
  \\ \hline
\end{longtable}


Hasil pengujian ini bisa dilihat pada tabel \ref{tb:gerakanputarrobot}.
Pada tabel tersebut, nilai yang ada di kolom \emph{speed} adalah besar kecepatan yang diatur pada \emph{topic} \lstinline{/cmd/vel},
  nilai yang ada di kolom \emph{estimated} didapatkan dari perkalian besar kecepatan putar dengan durasi pengujian,
  nilai yang ada di kolom \emph{measured} didapatkan dari pengukuran menggunakan kompas yang dipasang pada robot,
  dan terakhir nilai yang ada di kolom \emph{odometry} didapatkan dari data orientasi yang ada pada \emph{topic} \lstinline{/odom}.
Sama seperti pada tabel \ref{tb:gerakanputarsimulasi},
  Subkolom E pada kolom \emph{odometry} merupakan rata-rata \emph{error} pada nilai odometri terhadap nilai yang diharapkan (\emph{estimated}) dan nilai pengukuran (\emph{measured}) serta persentasenya jika dibandingkan dengan nilai sudut yang diharapkan.

Dari data yang dihasilkan oleh pengujian ini dapat diketahui bahwa gerakan putar yang diperintahkan kepada robot memiliki \emph{error} jarak sebesar 8.5-15.2 derajat dengan persentase rata-rata sebesar 7.8\% dari nilai sudut yang diharapkan.
Lebih lanjut, ketika hasil yang didapatkan ditampilkan sebagai grafik seperti yang terlihat pada gambar \ref{fig:grafikgerakanputarsimulasi},
  seperti kesimpulan pada pengujian sebelumnya,
  hasil yang didapatkan cenderung memiliki sudut dengan arah positif dan negatif yang sesuai dengan nilai yang diharapkan (\emph{expected}) dan nilai pengukuran (\emph{measured}).


\begin{figure}[ht]
  \centering
  \begin{tikzpicture}
    \begin{axis}[
        height=0.35\textwidth,
        width=0.9\textwidth,
        ylabel=Sudut (degree),
        xlabel=Percobaan Ke-,
        legend style={
          at={(0.5,1.5)},
          anchor=north,
          legend columns=-1,
        },
        ymajorgrids,
        bar width=5pt,
        ybar=0pt,
        xmin=0.1,
        xmax=6.9,
        ymin=0,
        xtick distance=1,
        ytick distance=90,
      ]
      \addplot table[x=index,y=expected,col sep=comma]{data/gerakan_putar_robot.csv};
      \addplot table[x=index,y=measured,col sep=comma]{data/gerakan_putar_robot.csv};
      \addplot table[x=index,y=odometry,col sep=comma]{data/gerakan_putar_robot.csv};
      \legend{Expected,Measured,Odometry}
    \end{axis}
  \end{tikzpicture}
  \caption{Grafik estimasi sudut orientasi akhir dari gerakan putar pada \emph{real robot}.}
  \label{fig:grafikgerakanputarrobot}
\end{figure}

