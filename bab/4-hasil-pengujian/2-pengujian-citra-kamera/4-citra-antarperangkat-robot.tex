\subsection{Pengujian Pengiriman Citra Kamera Antar-perangkat pada \emph{Real Robot}}
\label{subsec:citraantarperangkatrobot}

Pengujian pengiriman citra kamera antar-perangkat pada \emph{real robot} memiliki kesamaan proses dengan yang dilakukan di pengujian sebelumnya yang ada di bagian \ref{subsec:citraantarperangkatsimulasi},
  dimana di pengujian ini, \emph{image viewer node} dan \emph{command-line} \lstinline{$ ros2 topic} dijalankan di perangkat berbeda yang terhubung menggunakan jaringan \emph{ethernet} dengan perangkat yang menjalankan \emph{V4L2 camera node}.

\begin{longtable}{|c|c|c|c|c|c|}
  \caption{Hasil \emph{delay} dan frekuensi dari pengiriman citra kamera antar-perangkat pada \emph{real robot}.}
  \label{tb:pengirimancitraantarperangkatrobot}
  \\ \hline \rowcolor[HTML]{E0E0E0}
  \multicolumn{3}{|c|}{Resolution} &
  \multicolumn{1}{|c|}{Delay} &
  \multicolumn{2}{|c|}{Rate}
  \\ \hline \rowcolor[HTML]{E0E0E0}
  Width & Height & Size (KB) & ms & hz & percent
  \csvreader[head to column names]{data/pengiriman_citra_antarperangkat_robot.csv}{}{
    \\ \hline
    \width & \height & \size & \delay & \rate & \ratepercent
  }
  \\ \hline
\end{longtable}


Sama seperti pengujian-pengujian sebelumnya,
  pengujian ini juga dilakukan dengan berbagai macam konfigurasi resolusi citra yang dikirim.
Hasil pengujian ini bisa dilihat pada tabel \ref{tb:pengirimancitraantarperangkatrobot}.
Pada tabel tersebut \emph{width} dan \emph{height} merupakan resolusi citra yang dikirimkan,
  \emph{size} merupakan ukuran data yang dikirim,
  \emph{delay} merupakan selang waktu yang dibutuhkan untuk sampai ke perangkat penerima,
  dan \emph{rate} merupakan frekuensi pengiriman citra.


\begin{figure}[ht]
  \centering
  \begin{tikzpicture}
    \begin{axis}[
        height=0.3\textwidth,
        width=0.9\textwidth,
        ylabel=Delay (ms),
        ymajorgrids,
        ymin=0,
        ymax=125,
      ]
      \addplot table[x=size,y=delay,col sep=comma]{data/pengiriman_citra_antarperangkat_robot.csv};
    \end{axis}
  \end{tikzpicture}
  \begin{tikzpicture}
    \begin{axis}[
        height=0.3\textwidth,
        width=0.9\textwidth,
        xlabel=Data Size (KB),
        ylabel=Rate (hz),
        ymajorgrids,
        ymin=0,
        ymax=40,
      ]
      \addplot table[x=size,y=rate,col sep=comma]{data/pengiriman_citra_antarperangkat_robot.csv};
    \end{axis}
  \end{tikzpicture}
  \caption{Grafik \emph{delay} dan frekuensi dari pengiriman citra antar-perangkat pada \emph{real robot}.}
  \label{fig:grafikpengirimancitraantarperangkatrobot}
\end{figure}


Dari data yang dihasilkan oleh pengujian ini dapat diketahui bahwa data yang dikirimkan memiliki \emph{delay} sebesar 6-104 ms serta frekuensi sebesar 5-30 hz.
Lebih lanjut, ketika hasil yang didapatkan ditampilkan sebagai grafik seperti yang terlihat pada gambar \ref{fig:grafikpengirimancitraantarperangkatrobot},
  seperti kesimpulan pada pengujian sebelumnya,
  hasil yang didapatkan menunjukkan bahwa semakin besar ukuran data yang dikirimkan,
  maka \emph{delay} yang dihasilkan akan cenderung naik dan frekuensi yang dihasilkan akan cenderung turun.

Berbeda dengan yang dihasilkan di pengujian-pengujian sebelumnya,
  \emph{delay} yang dihasilkan selalu naik secara drastis,
  dimana pada resolusi 640 x 480, \emph{delay} yang dihasilkan bernilai 31.0 ms.
Sedangkan untuk frekuensi, hingga batas resolusi 640 x 480,
  pengiriman citra yang dilakukan bisa menghasilkan frekuensi di atas 90\% dari nilai FPS yang dimiliki kamera pada \emph{real robot}.
Namun, diatas resolusi tersebut, frekuensi yang dihasilkan akan turun drastis hingga 5.0 hz.
