\subsection{Pengujian Pengiriman Citra Kamera pada Robot di Simulasi}
\label{subsec:citrasimulasi}

\begin{figure}[ht]
  \centering
  \includegraphics[width=0.95\textwidth,keepaspectratio]{gambar/rosgraph-camera-plugin.png}
  \caption{Relasi antar-\emph{node} dari pengujian pengiriman citra kamera di simulasi.}
  \label{fig:rosgraphcameraplugin}
\end{figure}

Pengujian pengiriman citra kamera pada robot di simulasi dilakukan dengan cara menjalankan lingkungan \emph{indoor} pada simulator gazebo,
  menjalankan \emph{image viewer node} untuk melihat hasil pengiriman gambar,
  serta menjalankan \emph{command-line} \lstinline{$ ros2 topic delay} dan \lstinline{$ ros2 topic hz} untuk mengukur \emph{delay} serta frekuensi dari data yang dikirim.
Seperti yang terlihat pada gambar \ref{fig:rosgraphcameraplugin},
  \emph{node} \lstinline{/camera_plugin} akan mengirimkan \emph{topic} \lstinline{/camera/image_raw} yang berisi citra kamera,
  setelah itu \emph{node} \lstinline{/image_viewer} akan menerima citra tersebut dan menampilkannya dalam bentuk GUI.


\begin{figure}[ht]
  \centering
  \begin{tikzpicture}
    \begin{axis}[
        height=0.35\textwidth,
        width=0.9\textwidth,
        ylabel=Delay (ms),
        ymajorgrids,
        ymin=0,
        ymax=125,
      ]
      \addplot table[x=size,y=delay,col sep=comma]{data/pengiriman_citra_simulasi.csv};
    \end{axis}
  \end{tikzpicture}
  \begin{tikzpicture}
    \begin{axis}[
        height=0.35\textwidth,
        width=0.9\textwidth,
        xlabel=Data Size (KB),
        ylabel=Rate (hz),
        ymajorgrids,
        ymin=0,
        ymax=40,
      ]
      \addplot table[x=size,y=rate,col sep=comma]{data/pengiriman_citra_simulasi.csv};
    \end{axis}
  \end{tikzpicture}
  \caption{Grafik \emph{delay} dan frekuensi dari pengiriman citra pada robot di simulasi.}
  \label{fig:grafikpengirimancitrasimulasi}
\end{figure}


Pengujian ini dilakukan dengan berbagai macam konfigurasi resolusi citra yang dikirim.
Hasil pengujian ini bisa dilihat pada tabel \ref{tb:pengirimancitrasimulasi}.
Pada tabel tersebut \emph{width} dan \emph{height} merupakan resolusi citra yang dikirimkan,
  \emph{size} merupakan hasil perkalian resolusi dengan jumlah \emph{channel} (dalam hal ini 4 untuk citra RGBA),
  \emph{delay} merupakan selang waktu yang dibutuhkan sebelum citra sampai ke penerima,
  dan \emph{rate} merupakan frekuensi pengiriman citra.


\begin{figure}[ht]
  \centering
  \begin{tikzpicture}
    \begin{axis}[
        height=0.35\textwidth,
        width=0.9\textwidth,
        ylabel=Delay (ms),
        ymajorgrids,
        ymin=0,
        ymax=125,
      ]
      \addplot table[x=size,y=delay,col sep=comma]{data/pengiriman_citra_simulasi.csv};
    \end{axis}
  \end{tikzpicture}
  \begin{tikzpicture}
    \begin{axis}[
        height=0.35\textwidth,
        width=0.9\textwidth,
        xlabel=Data Size (KB),
        ylabel=Rate (hz),
        ymajorgrids,
        ymin=0,
        ymax=40,
      ]
      \addplot table[x=size,y=rate,col sep=comma]{data/pengiriman_citra_simulasi.csv};
    \end{axis}
  \end{tikzpicture}
  \caption{Grafik \emph{delay} dan frekuensi dari pengiriman citra pada robot di simulasi.}
  \label{fig:grafikpengirimancitrasimulasi}
\end{figure}


Dari data yang dihasilkan oleh pengujian ini dapat diketahui bahwa data yang dikirimkan memiliki \emph{delay} sebesar 0-11 ms serta frekuensi sebesar 16.5-29.8 hz.
Lebih lanjut, ketika hasil yang didapatkan ditampilkan sebagai grafik seperti yang terlihat pada gambar \ref{fig:grafikpengirimancitraantarperangkatsimulasi},
  hasil yang didapatkan menunjukkan bahwa semakin besar ukuran data yang dikirimkan,
  maka \emph{delay} yang dihasilkan akan cenderung naik dan frekuensi yang dihasilkan akan cenderung turun.
Walaupun cenderung naik, pada pengiriman dengan data ukuran terbesar,
  nilai \emph{delay} yang dihasilkan tersebut terhitung rendah, hanya sebesar 11 ms.
Sedangkan untuk frekuensi,
  hingga batas resolusi 800 x 600,
  pengiriman yang dilakukan bisa menghasilkan frekuensi diatas 90\% dari nilai FPS yang dimiliki kamera yang ada di simulasi.
