\subsection{Pengujian Pengiriman Citra Kamera Antar-perangkat pada Robot di Simulasi}
\label{subsec:citraantarperangkatsimulasi}

Pengujian pengiriman citra kamera antar-perangkat pada robot di simulasi memiliki kesamaan proses dengan yang dilakukan di pengujian pada sesama perangkat yang ada di bagian \ref{subsec:citrasimulasi}.
Hanya saja, di pengujian ini, \emph{image viewer node} dan \emph{command-line} \lstinline{$ ros2 topic} dijalankan di perangkat berbeda yang terhubung dengan perangkat yang menjalankan lingkungan simulasi menggunakan jaringan \emph{ethernet}.
Perangkat lain yang digunakan di pengujian ini adalah sebuah komputer Intel NUC dengan spesifikasi yang sama seperti yang dimiliki prototipe robot di tabel \ref{tb:spesifikasikomputerrobot}.

\begin{longtable}{|c|c|c|c|c|}
  \caption{Hasil \emph{delay} dan frekuensi dari pengiriman citra kamera antar-perangkat di simulasi.}
  \label{tb:pengirimancitraantarperangkatsimulasi}
  \\ \hline \rowcolor[HTML]{E0E0E0}
  Width & Height & Size (kb) & Delay (ms) & Rate (hz)
  \csvreader[head to column names]{data/pengiriman_citra_antarperangkat_simulasi.csv}{}{
    \\ \hline
    \width & \height & \size & \delay & \rate
  }
  \\ \hline
\end{longtable}


Sama seperti pengujian pada sesama perangkat,
  pengujian ini juga dilakukan dengan berbagai macam konfigurasi resolusi citra yang dikirim.
Hasil pengujian ini bisa dilihat pada tabel \ref{tb:pengirimancitraantarperangkatsimulasi}.
Pada tabel tersebut \emph{width} dan \emph{height} merupakan resolusi citra yang dikirimkan,
  \emph{size} merupakan ukuran data yang dikirim,
  \emph{delay} merupakan selang waktu yang dibutuhkan untuk sampai ke perangkat penerima,
  dan \emph{rate} merupakan frekuensi pengiriman citra.

Seperti yang dapat dilihat pada gambar \ref{fig:grafikpengirimancitraantarperangkatsimulasi},
  dari data yang dihasilkan oleh pengujian ini dapat diketahui bahwa semakin besar ukuran data yang dikirimkan,
  maka \emph{delay} akan cenderung naik dan frekuensi akan cenderung turun.
Pada pengujian ini, dapat diketahui bahwa frekuensi yang dihasilkan cenderung stabil ketika ukuran data yang dikirim ada di bawah 3000 KB,
  namun mengalami perubahan yang sangat signifikan ketika ukuran data yang dikirim ada di atas nilai tersebut.


\begin{figure} [ht]
  \centering
  \begin{tikzpicture}
    \begin{axis}[
        height=0.45\textwidth,
        width=0.9\textwidth,
        ylabel=Delay (ms),
        ymajorgrids,
        ymin=0,
        ymax=40,
      ]
      \addplot table[x=size,y=delay,col sep=comma]{data/pengiriman_citra_antarperangkat_simulasi.csv};
    \end{axis}
  \end{tikzpicture}
  \begin{tikzpicture}
    \begin{axis}[
        height=0.45\textwidth,
        width=0.9\textwidth,
        xlabel=Data Size (kb),
        ylabel=Rate (hz),
        ymajorgrids,
        ymin=0,
        ymax=40,
      ]
      \addplot table[x=size,y=rate,col sep=comma]{data/pengiriman_citra_antarperangkat_simulasi.csv};
    \end{axis}
  \end{tikzpicture}
  \caption{Grafik \emph{delay} dan frekuensi dari pengiriman citra antar-perangkat di simulasi.}
  \label{fig:grafikpengirimancitraantarperangkatsimulasi}
\end{figure}

