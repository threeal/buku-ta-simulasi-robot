\subsection{\emph{Gazebo Model}}
\label{subsec:gazebomodel}

\emph{Gazebo model} merupakan bagian dari simulator Gazebo yang merepresentasikan objek yang ada di lingkungan simulasi seperti robot, pengguna, ruangan, meja, kursi, dan lain sebagainya.
Selain \emph{Gazebo Model}, dikenal juga istilah \emph{Gazebo World} yang merepresentasikan keseluruhan lingkungan simulasi beserta model-model yang ada di dalamnya.

\lstinputlisting[
  language=XML,
  style=code,
  caption={Contoh \emph{file} SDFormat yang mendeskripsikan komponen dari suatu model.},
  label={lst:contohsdfmodel}
]{kode/sdf/model_example.xml}

\emph{Gazebo Model} dibentuk dari sebuah \emph{file} SDFormat,
  yang merupakan sebuah format berbasis XML untuk mendeskripsikan setiap komponen yang ada pada objek di lingkungan simulasi.
Pada Gazebo, SDFormat menggantikan format URDF yang digunakan untuk mendeskripsikan robot yang ada pada ROS.
Untuk mendeskripsikan sebuah model,
  seperti yang terlihat pada potongan kode \ref{lst:contohsdfmodel},
  model dari suatu objek akan terbagi menjadi beberapa \emph{link},
  yang mana pada setiap \emph{link} akan memiliki nilai inertia, massa, bentuk \emph{collision}, serta tampilan visual yang digunakan untuk mensimulasikan objek tersebut secara virtual.
