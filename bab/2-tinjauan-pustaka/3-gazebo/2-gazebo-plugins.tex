\subsection{\emph{Gazebo Plugin}}
\label{subsec:gazeboplugin}

\emph{Gazebo plugin} merupakan bagian dari simulator Gazebo yang memungkinkan dibuatnya program yang dapat memperoleh data maupun memanipulasi sistem yang ada di lingkungan simulasi \citep{url:gazeboplugin}.
\emph{Plugin} tersebut dapat dibuat menggunakan bahasa pemrograman C++ yang nantinya akan dikompilasi menjadi sebuah \emph{shared library} yang dapat diintegrasikan secara langsung pada simulator Gazebo.

\lstinputlisting[
  language=C++,
  style=code,
  caption={Contoh model \emph{plugin}.},
  label={lst:contohmodelplugin}
]{kode/plugin/plugin_example.cpp}

\emph{Gazebo plugin} terbagi menjadi beberapa jenis plugin,
  yakni \emph{world plugin}, \emph{model plugin}, \emph{sensor plugin}, \emph{system plugin}, \emph{visual plugin}, dan \emph{GUI plugin}.
Sebagai contoh seperti pada potongan kode \ref{lst:contohmodelplugin},
  sebuah \emph{plugin} dibuat dari sebuah \emph{class} yang diturunkan dari \emph{class} \lstinline{ModelPlugin}.
\emph{Class} tersebut memiliki sebuah fungsi \emph{constructor} yang dilakukan untuk menginisialisasi variabel yang dimiliki objek \emph{class} tersebut,
  dan sebuah fungsi \lstinline{Load()} yang digunakan untuk mendapatkan informasi dari \emph{element} yang ada pada SDFormat suatu model.
Agar bisa digunakan sebagai \emph{plugin} pada simulator Gazebo,
  \emph{class} yang dibuat nantinya perlu diregistrasi menggunakan \emph{macro} \lstinline{GZ_REGISTER_WORLD_PLUGIN}.
