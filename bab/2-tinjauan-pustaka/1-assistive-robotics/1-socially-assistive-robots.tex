\subsection{\emph{Socially Assistive Robots} (SARs)}
\label{subsec:sociallyassistiverobots}

\emph{Socially assistive robots} (SARs) merupakan jenis robot dalam bidang \emph{socially assistive robotics} yang menggabungkan aspek yang ada pada \emph{assistive robotics} dan \emph{socially interactive robotics}.
SARs memiliki tujuan yang sama dengan robot di bidang \emph{assistive robotics},
  yakni dalam hal memberikan bantuan kepada pengguna secara \emph{assistive}, namun pada SARs,
  bantuan tersebut secara spesifik diberikan melalui interaksi sosial kepada pengguna.
Karena adanya aspek interaksi sosial tersebut, SARs memiliki tujuan yang sama dengan robot di bidang \emph{socially interactive robotics}.

Rich dan Sidner \citep{cit:rich2009} memaparkan,
  SARs mampu memberikan bantuan kepada pengguna dalam berbagai cakupan.
Cakupan tersebut terdiri atas kemampuan SARs untuk memberikan dukungan fungsional dan kognitif kepada pengguna,
  memberikan kesempatan bagi pengguna untuk meningkatkan partisipasi sosial dan kesehatan psikologis,
  menyediakan pemantauan jarak jauh dan berkelanjutan atas status kesehatan pengguna,
  serta memfasilitasi pengguna untuk melakukan perilaku hidup sehat dan pencapaian tujuan yang berhubungan dengan kesehatan.
