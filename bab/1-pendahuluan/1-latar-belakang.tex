\section{Latar Belakang}
\label{sec:latarbelakang}

Selama beberapa tahun terakhir, robot telah mengalami perkembangan yang signifikan dari robot beroda untuk edukasi \citep{cit:goncalves2009} hingga robot manipulator untuk skala industri \citep{cit:blatnicky2020}.
Salah satu bentuk perkembangan lain dari robot tersebut adalah \emph{socially assistive robots} (SARs).
SARs merupakan jenis robot dalam bidang \emph{socially assistive robotics} yang menggabungkan aspek yang ada pada \emph{assistive robotics} dan \emph{socially interactive robotics} sehingga menjadikan SARs sebagai robot yang mampu memberikan bantuan kepada pengguna dalam bentuk interaksi sosial \citep{cit:seifer2005}.

Namun, karena sifat dari SARs yang melibatkan interaksi langsung dengan pengguna, maka pengujian dari robot akan menjadi sulit dan beresiko bagi pengguna yang ikut terlibat dalam pengujian tersebut \citep{cit:erickson2020}.
Salah satu solusi untuk mengatasi masalah tersebut adalah dengan melakukan pengujian secara virtual melalui simulasi.
Selain bisa meminimalisir resiko, penggunaan simulasi juga bisa mengurangi biaya yang dibutuhkan dan menghemat waktu pengujian selama pengembangan robot tersebut \citep{cit:takaya2016}.

Hingga saat ini sudah ada beberapa simulator yang bisa digunakan untuk menjalankan simulasi robot seperti Webots \citep{cit:michel2004}, Gazebo \citep{cit:koenig2004}, V-REP \citep{cit:rohmer2013}, OpenAI Gym \citep{cit:brockman2016}, dan lain sebagainya.
Namun, simulator-simulator tersebut hanyalah platform yang secara umum digunakan untuk membantu pengembangan robot melalui simulasi virtual.
Sedangkan pengembangan dari lingkungan simulasi dan kontroler robot untuk simulasi tersebut harus dibuat sendiri oleh pengembang robot.

Untuk itu, pada penelitian ini kami mengajukan penelitian terkait pengembangan lingkungan simulasi untuk pengujian SARs menggunakan ROS 2 dan Gazebo.
ROS 2 dan Gazebo sendiri dipilih karena tersedianya banyak library yang dapat membantu pengembangan maupun pengujian robot, terutama untuk simulasi.
Selain itu, dengan adanya ROS 2, kontroler robot yang diuji melalui simulasi bisa dengan mudah dipindahkan ke robot fisik untuk diuji secara langsung pada pengguna \citep{cit:takaya2016}.
