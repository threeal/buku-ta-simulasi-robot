\section{Sistematika Penulisan}
\label{sec:sistematikapenulisan}

Buku penelitian tugas akhir ini disusun dalam sistematika yang terstruktur agar mudah dipahami dan dipelajari oleh pembaca.
Bab \ref{chap:pendahuluan} pada buku ini berisi uraian tentang latar belakang, permasalahan yang diangkat, penelitian terkait, tujuan, serta batasan masalah dari penelitian.
Kemudian bab \ref{chap:tinjauanpustaka} menguraikan teori-teori penunjang yang berhubungan dengan penelitian ini,
  seperti uraian tentang \emph{assistive robotics}, simulator Gazebo, Robot Operating System (ROS),
  dan lain sebagainya.
Pada bab \ref{chap:desainimplementasi}, desain dan implementasi dari sistem yang dibuat diuraikan,
  dari perancangan model robot dan pengguna untuk simulasi, pengembangan lingkungan simulasi,
  dan integrasi sistem kontroler pada simulasi dan pada robot fisik.
Lalu pada bab \ref{chap:hasilpengujian}, dijelaskan pengujian yang dilakukan dari sistem yang sudah dibuat dan hasil yang didapatkan.
pengujian ini dilakukan dengan beberapa cara seperti pengujian terhadap gerakan robot,
  tangkapan citra kamera, pemetaan ruangan, dan lain sebagainya.
Terakhir, kesimpulan dari penelitian serta saran untuk penelitian berikutnya diuraikan pada bab \ref{chap:penutup} yang ada di buku ini.
