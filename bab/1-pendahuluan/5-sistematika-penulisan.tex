\section{Sistematika Penulisan}
\label{sec:sistematikapenulisan}

Laporan penelitian tugas akhir ini tersusun dalam sistematika dan terstruktur sehingga mudah dipahami dan dipelajari oleh pembaca maupun seseorang yang ingin melanjutkan penelitian ini.
Alur sistematika penulisan laporan penelitian ini yaitu:

\begin{enumerate}[nolistsep]

  \item \textbf{BAB I Pendahuluan}

  Bab ini berisi uraian tentang latar belakang permasalahan, penegasan dan alasan pemilihan judul, sistematika laporan, tujuan, dan metodologi penelitian.

  \vspace{2ex}

  \item \textbf{BAB II Tinjauan Pustaka}

  Bab ini berisi tentang uraian secara sistematis teori-teori yang berhubungan dengan permasalahan yang dibahas pada penelitian ini.
  Teori-teori ini digunakan sebagai dasar dalam penelitian, yaitu informasi terkait \textcolor{red}{\lipsum[1][1-2]}

  \vspace{2ex}

  \item \textbf{BAB III Desain dan Implementasi Sistem}

  Bab ini berisi tentang penjelasan-penjelasan terkait eksperimen yang akan dilakukan, \textcolor{red}{\lipsum[1][1-3]}

  \vspace{2ex}

  \item \textbf{BAB IV Pengujian dan Analisa}

  Bab ini menjelaskan tentang hasil serta analisis yang didapatkan dari pengujian yang dilakukan mulai dari hasil pengujian \textcolor{red}{\lipsum[1][1-2]}

  \vspace{2ex}

  \item \textbf{BAB V Penutup}

  Bab ini merupakan penutup yang berisi kesimpulan yang diambil dari penelitian dan pengujian yang telah dilakukan.
  Saran dan kritik yang membangun untuk pengembangan lebih lanjut juga dituliskan pada bab ini.

\end{enumerate}
