\section{Kesimpulan}
\label{sec:kesimpulan}

Berdasarkan hasil pengujian yang telah dilakukan,
  dapat disimpulkan bahwa lingkungan simulasi yang dibuat pada simulator Gazebo dapat digunakan untuk melakukan pengujian pada SARs secara virtual.
Hasil pembacaan data sensor pada robot, seperti kamera dan \emph{depth camera},
  dapat disimulasikan pada lingkungan yang dibuat sehingga memungkinkan pengambilan data pengujian untuk dilakukan secara virtual.
Selain itu, dengan adanya lingkungan virtual,
  pengujian dapat dilakukan pada berbagai macam kondisi lingkungan yang diinginkan tanpa perlu mengeluarkan biaya tambahan untuk membeli perabotan maupun merenovasi ruangan seperti yang perlu dilakukan pada pengujian \emph{real}.

Kemudian, Dengan adanya abstraksi dari sistem kontroler yang dibuat menggunakan ROS 2,
  program \emph{behavior} yang diujikan pada model robot di simulasi memiliki hasil yang sama ketika diujikan pada robot fisik.
Selama pengujian, terdapat beberapa program yang sudah tersedia di ROS 2 sehingga pengujian dapat dilakukan dengan mudah tanpa perlu menulis keseluruhan program yang diperlukan untuk melakukan pengujian.
Lebih lanjut, dengan performa yang dimiliki oleh ROS 2,
  pertukaran data yang terjadi antara program yang terabstraksi dapat dilakukan secara \emph{real-time} dengan \emph{delay} yang rendah.
