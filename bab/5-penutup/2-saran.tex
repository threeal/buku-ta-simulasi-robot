\section{Saran}
\label{chap:saran}

Pengembangan lebih lanjut dari penelitian ini bisa dilakukan dengan meningkatkan keakuratan komponen yang ada serta penambahan komponen yang bisa digunakan seperti lengan \emph{manipulator} dan \emph{lidar scanner} pada model robot yang ada di simulasi.
Pilihan jenis perangkat yang digunakan pada robot fisik juga bisa ditambahkan,
  seperti Intel RealSense dan ZED \emph{stereo camera} sebagai pengganti Kinect V2 untuk komponen \emph{depth camera}.

Untuk simulator, sebagai alternatif dari Gazebo,
  beberapa pilihan simulator lain juga bisa digunakan untuk melakukan pengujian seperti Webots dan OpenAI Gym.
Selain itu, untuk mendapatkan hasil visual yang lebih baik,
  simulator juga bisa dikembangkan menggunakan \emph{game engine} terkini yang mampu mensimulasikan lingkungan 3D beserta \emph{physics} yang ada di dalamnya seperti pada \emph{game engine} Unity dan Unreal.

Terkait dengan \emph{assistive robotics},
  pengembangan lebih lanjut juga bisa dilakukan dengan meningkatkan interaksi antara manusia dengan robot,
  seperti membawa pengguna  untuk berinteraksi dengan robot di dunia simulasi menggunakan perangkat \emph{virtual reality} (VR) maupun mensimulasikan masukan suara dari pengguna yang dapat ditangkap oleh robot yang ada di simulasi.
Lebih lanjut, pengembangan juga bisa dilakukan dengan meningkatkan model pengguna yang ada di simulasi, seperti memberikan kemampuan untuk mengubah ekspresi maupun meningkatkan animasi gerakan yang dilakukan.
