\subsection{\emph{Move For Node}}
\label{subsec:movefornode}

\emph{Move for node} merupakan \emph{behavior node} yang digunakan untuk memberikan perintah gerakan linier dan gerakan putar selama kurun waktu yang ditentukan.
\emph{Node} ini ditulis dalam bahasa pemrograman C++ dan akan mengirimkan data gerakan kecepatan pada topic \lstinline{/cmd_vel} secara terus menerus hingga durasi waktu yang ditentukan sudah terpenuhi.

\lstinputlisting[
  language=C++,
  style=code,
  caption={Program \emph{move for node}.},
  label={lst:movefornode}
]{kode/node/move_for_node.cpp}

Seperti yang terlihat pada potongan kode \ref{lst:movefornode},
  \emph{node} ini menggunakan \emph{package} \lstinline{rclcpp} untuk mengakses sistem komunikasi ROS 2 pada program yang ditulis menggunakan bahasa pemrograman C++.
\emph{Node} ini memiliki sebuah \emph{twist publisher} yang akan mengirimkan data perintah kecepatan gerak pada \emph{topic} \lstinline{/cmd_vel}.
Agar durasi proses bisa tersinkronisasi dengan waktu yang ada di simulasi,
  \emph{node} ini akan diproses sesuai dengan \emph{clock topic} yang diterima dari simulasi.
Selain itu, \emph{node} ini akan diproses sesuai dengan \emph{update timer} yang dimiliki \emph{node} tersebut.
