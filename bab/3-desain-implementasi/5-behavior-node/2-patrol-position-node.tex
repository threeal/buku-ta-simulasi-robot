\subsection{\emph{Patrol Position Node}}
\label{subsec:patrolpositionnode}

\emph{Patrol position node} merupakan \emph{behavior node} yang digunakan untuk memberikan perintah pada robot untuk bergerak menuju setiap posisi yang ditentukan sampai posisi terakhir.
\emph{Node} ini ditulis dalam bahasa pemrograman C++ dan akan menerima data odometri melalui \emph{topic} \lstinline{/odom},
  kemudian memproses target kecepatan yang diperlukan,
  dan mengirimnya melalui \emph{topic} \lstinline{/cmd_vel}.

\lstinputlisting[
  language=C++,
  style=code,
  caption={Program \emph{patrol position node}.},
  label={lst:patrolpositionnode}
]{kode/node/patrol_position_node.cpp}

Seperti yang terlihat pada potongan kode \ref{lst:patrolpositionnode},
  \emph{node} ini menggunakan \emph{package} \lstinline{rclcpp} untuk mengakses sistem komunikasi ROS 2 pada program yang ditulis menggunakan bahasa pemrograman C++.
\emph{Node} ini memiliki sebuah \emph{odometry subscription} untuk menerima data odometri pada \emph{topic} \lstinline{/odom} dan sebuah \emph{twist publisher} yang akan mengirimkan data perintah kecepatan gerak pada \emph{topic} \lstinline{/cmd_vel}.
Kemudian proses dari \emph{node} ini akan dilakukan secara terus menerus sampai posisi terakhir tercapai melalui \emph{callback} yang diberikan pada \emph{update timer}.
