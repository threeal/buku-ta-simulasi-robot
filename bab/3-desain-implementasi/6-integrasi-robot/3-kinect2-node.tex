\subsection{\emph{Kinect2 Node}}
\label{subsec:kinect2node}

\emph{Kinect2 node} merupakan \emph{node} yang digunakan untuk mengakses perangkat \emph{depth camera} Kinect V2 yang dimiliki robot \emph{Dienen}.
\emph{Node} ini ditulis dalam bahasa pemrograman C++ dan menggunakan \emph{library} \lstinline{libfreenect2} untuk memungkinkan akses terhadap perangkat Kinect V2 \citep{sft:libfreenect2}.
\emph{Node} ini akan mengirimkan sebuah data citra berwarna melalui \emph{topic} \lstinline{/kinect2/image_raw},
  sebuah data citra kedalaman (\emph{depth image}) melalui \emph{topic} \lstinline{/kinect2/depth/image_raw},
  dan dua buah data informasi kamera masing-masing untuk citra berwarna dan kedalaman.

Seperti yang terlihat pada potongan kode \ref{lst:kinect2node},
  \emph{node} ini menggunakan \emph{package} \lstinline{rclcpp} untuk mengakses sistem komunikasi ROS 2 dan objek \lstinline{libfreenect2::Freenect2} untuk mengakses perangkat Kinect V2.
\emph{Node} ini memiliki dua buah \emph{image publisher} dan \emph{camera info publisher} masing-masing untuk citra berwarna dan citra kedalaman.
Bagian utama dari \emph{node} ini terletak pada fungsi \lstinline{OnNewFrame()} yang akan dipanggil ketika \emph{node} menerima data citra dari perangkat Kinect V2,
  dan kemudian menyalurkan data citra tersebut kepada \emph{topic} yang telah dijelaskan sebelumnya.

\lstinputlisting[
  language=C++,
  style=code,
  caption={\emph{Class} dari \emph{Kinect2 node}.},
  label={lst:kinect2node}
]{kode/node/kinect2_node.cpp}
