\subsection{\emph{Navigation Plugin}}
\label{subsec:navigationplugin}

\emph{Navigation plugin} merupakan \emph{Gazebo plugin} yang digunakan untuk mengabstraksi komponen navigasi di simulasi.
\emph{Plugin} ini ditulis dalam bahasa C++ dan dibuat menjadi \emph{class} dengan nama \lstinline{dienen_gazebo_plugins::NavigationPlugin}.
Seperti yang terlihat pada potongan kode \ref{lst:navigationplugin},
  \emph{class} ini dibuat dengan menurunkan \emph{class} \lstinline{gazebo::ModelPlugin} sebagai \emph{parent class} dari \emph{class} ini sehingga memungkinkan \emph{class} ini untuk mengakses maupun memanipulasi data yang ada pada \emph{Gazebo model}.

\emph{Plugin} ini digunakan untuk mengirimkan estimasi posisi dan orientasi dari robot sebagai odometri melalui \emph{topic} \lstinline{/odom} dan menerima masukan kecepatan (\emph{twist}) melalui \emph{topic} \lstinline{/cmd_vel}.
Odometri yang dikirimkan didapatkan dari posisi dan orientasi mutlak dari model robot yang ada di simulasi,
  sedangkan masukan kecepatan yang diterima akan diterjemahkan sebagai memberikan \emph{force} kepada model robot yang mengakibatkan model tersebut bergerak ke arah yang sesuai dengan \emph{force} yang diberikan.

\lstinputlisting[
  language=C++,
  style=code,
  caption={\emph{Class} dari \emph{navigation plugin}.},
  label={lst:navigationplugin}
]{kode/plugin/navigation_plugin.cpp}

Setelah \emph{plugin} dibuat,
  file SDFormat dari model robot perlu diubah dengan menyematkan \emph{plugin element} di file tersebut.
Seperti yang terlihat pada potongan kode \ref{lst:integrasinavigationplugin},
  sebuah \emph{plugin element} ditambahkan di model robot dengan \emph{name attribute} berupa nama \emph{plugin} sekaligus nama \emph{node} yang akan digunakan dan \emph{filename attribute} yang merujuk pada nama \emph{shared library} dari \emph{plugin} yang telah dikompilasi.

\lstinputlisting[
  language=XML,
  style=code,
  caption={Integrasi \emph{navigation plugin} pada model robot.},
  label={lst:integrasinavigationplugin}
]{kode/sdf/plugin/navigation_plugin.xml}
