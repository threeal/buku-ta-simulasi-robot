\section{Pengembangan Model Pengguna}
\label{sec:modelpengguna}

Model pengguna yang ada di simulasi akan dibuat menyerupai bentuk manusia yang dapat berinteraksi dengan robot.
Pada kondisi pengujian menggunakan robot \emph{real},
  pengguna akan berinteraksi dengan robot melalui \emph{smart assistive posture device} yang digunakan untuk mengirimkan data ke robot yang berupa koordinat posisi dan orientasi pengguna,
  masukan suara yang diberikan pengguna melalui \emph{voice recognition},
  dan kondisi postur kaki dari pengguna, apakah sedang berdiri atau duduk.
Dengan adanya model pengguna ini,
  pengujian yang ada di simulasi dapat dilakukan terhadap model pengguna dengan perilaku yang sudah diprogram di awal,
  serta terhadap \emph{real user} yang berinteraksi dengan model robot di simulasi melalui data yang dikirim dari \emph{smart assistive posture device}.

Pada simulator Gazebo terdapat objek \emph{Gazebo Actors} yang dapat digunakan untuk menampilkan model manusia yang dapat bergerak dan melakukan animasi seperti berjalan, duduk, dan lain sebagainya.
Namun, objek tersebut sulit untuk diubah animasinya ketika simulasi sudah berjalan,
  sehingga kondisi perubahan postur kaki dari pengguna tidak bisa ditampilkan di model yang ada di simulasi.
Oleh karena itu, pada penelitian ini model pengguna dibuat dengan cara yang sama dengan model robot dengan memisahkan setiap bagian pengguna yang dapat bergerak bebas menjadi \emph{link element} dan kemudian akan saling terhubung menggunakan \emph{joint element}.

\subimport{2-model-pengguna}{1-sdformat-pengguna.tex}
