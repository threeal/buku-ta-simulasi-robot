\section{Pengembangan \emph{Behavior Node}}
\label{sec:behaviornode}

\emph{Behavior node} merupakan \emph{ROS 2 node} yang dibuat untuk mengatur tindakan yang akan dilakukan oleh SARs berdasarkan data yang diterima dari \emph{node} lain.
Seperti yang dijelaskan di bagian \ref{sec:integrasiplugin},
  \emph{behavior node} akan terhubung dengan \emph{node} lain yang secara abstrak merepresentasikan komponen yang dimiliki oleh robot \emph{Dienen} seperti penjelasan di bagian \ref{sec:modelrobot}.

\emph{Behavior node} dijalankan secara terpisah dari \emph{node} yang mengatur komponen yang ada di robot seperti komponen kamera, \emph{depth camera}, \emph{manipulator}, dan navigasi.
Terpisahnya \emph{behavior node} ini dilakukan untuk membuat pengujian pada SARs bisa dilakukan menggunakan berbagai macam data yang diterima,
  baik itu data \emph{real-time} yang berasal dari \emph{real robot},
  data yang direkam dari robot,
  maupun data yang ada di simulasi.

\subimport{5-behavior-node}{1-move-for-node.tex}
\subimport{5-behavior-node}{2-patrol-position-node.tex}
% \subimport{5-behavior-node}{3-image-viewer-node.tex}
% \subimport{5-behavior-node}{4-legs-teleop-node.tex}
% \subimport{5-behavior-node}{5-pose-detector-node.tex}
% \subimport{5-behavior-node}{6-rtabmap-node.tex}
