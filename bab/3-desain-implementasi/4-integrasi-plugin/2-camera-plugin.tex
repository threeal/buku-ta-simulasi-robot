\subsection{\emph{Camera Plugin}}
\label{subsec:cameraplugin}

\emph{Camera plugin} merupakan \emph{Gazebo plugin} yang digunakan untuk mengabstraksi komponen kamera di simulasi.
\emph{Plugin} yang digunakan ini akan mengirimkan data citra melalui \emph{topic} \lstinline{/image_raw} dan mengirimkan informasi kamera seperti resolusi citra, FoV, matriks proyeksi, dan lain sebagainya melalui \emph{topic} \lstinline{/camera_info}.
\emph{Plugin} ini didapatkan dari \emph{package} \lstinline{gazebo_ros},
  sehingga yang perlu dilakukan hanyalah mengintegrasikan \emph{plugin} tersebut ke model robot yang sudah dibuat di bagian \ref{sec:modelrobot}.

\lstinputlisting[
  language=XML,
  style=code,
  caption={Integrasi \emph{camera plugin} pada model robot.},
  label={lst:integrasicameraplugin}
]{kode/sdf/plugin/camera_plugin.xml}

Seperti yang terlihat pada potongan kode \ref{lst:integrasicameraplugin},
  sebuah \emph{plugin element} ditambahkan sebagai \emph{child element} dari \emph{sensor element}.
\emph{Plugin} ini menggunakan \emph{name attribute} untuk menentukan nama node dan \emph{filename attribute} untuk menentukan \emph{shared library} dari \emph{plugin} yang telah dikompilasi.
Dalam hal ini \emph{plugin} yang digunakan adalah \emph{plugin} \lstinline{gazebo_ros_camera} dari \emph{package} \lstinline{gazebo_ros}.
Sebagai tambahan, \emph{plugin} ini menggunakan \emph{camera name element} untuk menentukan \emph{namespace} dari \emph{topic} yang digunakan,
  sehingga dengan \emph{camera name element} bernilai \emph{camera} maka \emph{topic} yang digunakan adalah \lstinline{/camera/image_raw} dan \lstinline{/camera/camera_info}.
