\subsection{Struktur SDFormat Robot}
\label{subsec:sdformatrobot}

Dari paparan desain yang dijelaskan di bagian \ref{subsec:desainrobotdienen} dan paparan spesifikasi dari prototipe robot yang dijelaskan di bagian \ref{subsec:prototiperobotdienen},
  model robot untuk simulasi kemudian dibuat menggunakan model 3D dari desain yang ada dan disusun berdasarkan bagian dan komponen yang ada pada desain tersebut.

Penyusunan SDFormat dilakukan dengan memisahkan bagian yang dapat bergerak secara bebas di robot menjadi \emph{link element} secara terpisah.
Untuk setiap \emph{link element} yang ada,
  beberapa \emph{element} lain perlu ditambahkan ke \emph{element} tersebut sehingga model yang dibuat bisa digunakan di simulasi.
Seperti yang terlihat pada potongan kode \ref{lst:headlink},
  setiap \emph{link element} perlu memiliki \emph{pose element} untuk menentukan posisi dari bagian robot yang diukur dari titik pusat robot,
  \emph{inertial element} untuk menentukan berat dari bagian robot,
  \emph{collision element} untuk menentukan \emph{bounding box} yang digunakan oleh bagian robot,
  dan \emph{visual element} yang digunakan untuk menentukan tampilan dari bagian robot tersebut.

\lstinputlisting[
  language=XML,
  caption={\emph{Link element} untuk bagian kepala robot.},
  label={lst:headlink}
]{kode/sdf/robot_model/head_link.xml}

Agar terhubung satu sama lain,
  sebuah \emph{joint element} perlu ditambahkan untuk menentukan hubungan antara dua \emph{element}.
Seperti yang terlihat pada potongan kode \ref{lst:leftshoulderjoint} dan \ref{lst:upperbodyjoint},
  setiap \emph{joint element} memiliki \emph{type attribute} yang bisa bernilai \emph{fixed} maupun \emph{revolute}.
\emph{Joint element} dengan \emph{type attribute} bernilai \emph{fixed} digunakan untuk bagian yang tidak bisa bergerak bebas namun secara model terpisah seperti bagian bawah robot dan bagian atas robot,
  sedangkan \emph{joint element} dengan \emph{type attribute} bernilai \emph{revolute} digunakan untuk bagian yang bisa bergerak berputar pada satu sumbu, seperti bagian \emph{joint} yang ada di tangan.

\lstinputlisting[
  language=XML,
  caption={\emph{Joint element} yang menghubungkan bagian pundak kiri dan bagian badan atas robot.},
  label={lst:leftshoulderjoint}
]{kode/sdf/robot_model/left_shoulder_joint.xml}

\lstinputlisting[
  language=XML,
  caption={\emph{Joint element} yang menghubungkan bagian badan atas dan bagian badan bawah robot.},
  label={lst:upperbodyjoint}
]{kode/sdf/robot_model/upper_body_joint.xml}

\lstinputlisting[
  language=XML,
  caption={\emph{Sensor element} dari sensor kamera.},
  label={lst:camerasensor}
]{kode/sdf/robot_model/camera_sensor.xml}

Terakhir, penyusunan SDFormat dilakukan dengan menambahkan sensor kamera dan \emph{depth camera} yang digunakan pada robot.
Sensor tersebut ditambahkan dengan menyematkan \emph{sensor element} pada \emph{link element} dari masing-masing bagian kamera dan \emph{depth camera}.
Seperti yang terlihat pada potongan kode \ref{lst:camerasensor} dan \ref{lst:depthcamerasensor},
  sensor kamera dan \emph{depth camera} memiliki jumlah dan jenis \emph{child element} yang sama,
  hanya berbeda di beberapa nilai dari setiap \emph{element} dan nilai \emph{type attribute} yang digunakan.
  Pada kamera, \emph{sensor element} yang digunakan memiliki \emph{type attribute} bernilai \emph{camera},
  sedangkan pada \emph{depth camera}, \emph{sensor element} yang digunakan memiliki \emph{type attribute} bernilai \emph{depth}.
  Sisanya, nilai setiap \emph{element} yang ada di kedua sensor tersebut didapatkan dari spesifikasi kamera dan \emph{depth camera} yang digunakan di prototipe robot seperti yang dijelaskan di bagian \ref{subsec:prototiperobotdienen}.

\lstinputlisting[
  language=XML,
  caption={\emph{Sensor element} dari sensor \emph{depth camera}.},
  label={lst:depthcamerasensor}
]{kode/sdf/robot_model/depth_camera_sensor.xml}
