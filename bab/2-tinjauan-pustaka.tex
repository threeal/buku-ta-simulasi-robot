\chapter{TINJAUAN PUSTAKA}
\label{chap:tinjauanpustaka}

Pada bab ini akan dijelaskan teori-teori penunjang yang digunakan sebagai bahan acuan dan referensi untuk penelitian yang dilakukan.
Teori-teori yang dijelaskan pada bab ini akan dipaparkan dalam urutan yang sistematis,
  dimulai dari hal paling mendasar yang digunakan pada penelitian ini seperti penjelasan mengenai \emph{assistive robotics}, Gazebo, dan ROS,
  hingga penjelasan lebih dalam yang berhubungan dengan sistem yang dibuat dan pengujian yang dilakukan seperti Kinect dan SLAM.

\subimport{2-tinjauan-pustaka}{1-assistive-robotics.tex}
\subimport{2-tinjauan-pustaka}{2-gazebo.tex}
% \subimport{2-tinjauan-pustaka}{3-sdformat.tex}
\subimport{2-tinjauan-pustaka}{4-ros2.tex}
% \subimport{2-tinjauan-pustaka}{5-holonomic-robot.tex}
% \subimport{2-tinjauan-pustaka}{6-imu.tex}
% \subimport{2-tinjauan-pustaka}{7-assistive-device.tex}
% \subimport{2-tinjauan-pustaka}{8-depth-camera.tex}
% \subimport{2-tinjauan-pustaka}{9-kinect-v2.tex}
% \subimport{2-tinjauan-pustaka}{10-mediapipe.tex}
% \subimport{2-tinjauan-pustaka}{11-slam.tex}
