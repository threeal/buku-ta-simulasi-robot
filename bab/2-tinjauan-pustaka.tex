\chapter{TINJAUAN PUSTAKA}
\label{chap:tinjauanpustaka}

Demi mendukung penelitian ini, dibutuhkan beberapa teori penunjang sebagai bahan acuan dan referensi.
Dengan demikian penelitian ini menjadi lebih terarah.

\subimport{2-tinjauan-pustaka}{1-assistive-robotics.tex}
\subimport{2-tinjauan-pustaka}{2-gazebo.tex}
\subimport{2-tinjauan-pustaka}{3-sdformat.tex}
\subimport{2-tinjauan-pustaka}{4-ros2.tex}
\subimport{2-tinjauan-pustaka}{5-holonomic-robot.tex}
\subimport{2-tinjauan-pustaka}{6-imu.tex}
\subimport{2-tinjauan-pustaka}{7-assistive-device.tex}
\subimport{2-tinjauan-pustaka}{8-depth-camera.tex}
\subimport{2-tinjauan-pustaka}{9-kinect-v2.tex}
\subimport{2-tinjauan-pustaka}{10-mediapipe.tex}
\subimport{2-tinjauan-pustaka}{11-slam.tex}
