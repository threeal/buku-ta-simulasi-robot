\begin{sidewaystable}
  \centering
  \caption{Hasil estimasi posisi dari gerakan linier pada robot di simulasi selama 3 detik.}
  \label{tb:gerakanliniersimulasi}
  \begin{tabular}{|c|c|c|c|c|c|c|c|c|}
    \hline \rowcolor[HTML]{E0E0E0}
    \multicolumn{2}{|c|}{\textbf{Speed}} &
    \multicolumn{2}{|c|}{\textbf{Expected Position}} &
    \multicolumn{2}{|c|}{\textbf{Measured Position}} &
    \multicolumn{3}{|c|}{\textbf{Odometry Position}}
    \\ \hline \rowcolor[HTML]{E0E0E0}
    \textbf{X (m/s)} & \textbf{Y (m/s)} &
    \textbf{X (m)} & \textbf{Y (m)} &
    \textbf{X (m)} & \textbf{Y (m)} &
    \textbf{X (m)} & \textbf{Y (m)} & \textbf{Error}
    \csvreader[head to column names]{data/gerakan_linier_simulasi.csv}{}{
      \\ \hline
      \speedx & \speedy &
      \expectedx & \expectedy &
      \measuredx & \measuredy &
      \odometryx & \odometryy & \odometryerror
    }
    \\ \hline
  \end{tabular}
\end{sidewaystable}
