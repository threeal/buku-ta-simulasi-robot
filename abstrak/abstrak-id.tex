\begin{center}
  \large\textbf{ABSTRAK}
\end{center}

\vspace{2ex}

\begingroup
  \setlength{\tabcolsep}{0pt}
  \noindent
  \begin{tabularx}{\textwidth}{l >{\centering}m{2em} X}
    Nama        &:& Muhammad Alfi Maulana Fikri \\
    Judul       &:&	Pengembangan Lingkungan Simulasi untuk Pengujian \emph{Socially Assistive Robots} Menggunakan ROS 2 dan Gazebo \\
    Pembimbing  &:& 1. Prof. Dr. Ir. Mauridhi Hery Purnomo, M.Eng. \\
                & & 2. Dr. I Ketut Eddy Purnama, S.T., M.T. \\
  \end{tabularx}
\endgroup

Pada penelitian ini kami mengajukan lingkungan simulasi yang dikembangkan menggunakan ROS 2 dan Gazebo untuk pengujian \emph{socially assistive robots} (SARs).
Di dalam lingkungan simulasi ini, model robot yang digunakan akan diujikan dengan model pengguna serta model-model objek lain secara virtual.
Untuk mempermudah pemindahan program dari simulasi ke robot fisik, kontroler robot akan dikembangkan secara terpisah dari lingkungan simulasi yang mana ketika pengujian, keduanya akan saling terhubung menggunakan sistem komunikasi antar proses yang ada di ROS 2.
Diharapkan, lingkungan simulasi yang dibuat dapat membantu pengujian SARs dengan meminimalisir resiko, mengurangi biaya, dan menghemat waktu jika dibandingkan dengan melakukan pengujian secara langsung menggunakan robot fisik.

Kata Kunci: Simulasi, \emph{Assistive Robotics}, ROS2, Gazebo.
