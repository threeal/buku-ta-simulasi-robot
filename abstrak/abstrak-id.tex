\begin{center}
  \large\textbf{ABSTRAK}
\end{center}

\vspace{2ex}

\begingroup
  \setlength{\tabcolsep}{0pt}
  \noindent
  \begin{tabularx}{\textwidth}{l >{\centering}m{2em} X}
    Nama        &:& Muhammad Alfi Maulana Fikri \\
    Judul       &:&	Pengembangan Lingkungan Simulasi untuk Pengujian \emph{Socially Assistive Robots} Menggunakan ROS 2 dan Gazebo \\
    Pembimbing  &:& 1. Prof. Dr. Ir. Mauridhi Hery Purnomo, M.Eng. \\
                & & 2. Dr. I Ketut Eddy Purnama, S.T., M.T. \\
  \end{tabularx}
\endgroup

Selama beberapa tahun terakhir,
  robot telah mengalami perkembangan yang cukup signifikan.
Salah satu bentuk perkembangan tersebut adalah \emph{socially assistive robots} (SARs) yang mampu memberikan bantuan kepada pengguna dalam bentuk interaksi sosial.
Namun,
  karena sifatnya yang melibatkan interaksi langsung dengan pengguna,
  pengujian pada SARs akan menjadi sulit dan beresiko.
Untuk itu,
  pada penelitian ini kami mengajukan lingkungan simulasi untuk pengujian SARs yang dibuat menggunakan simulator Gazebo.
Di dalam lingkungan simulasi ini,
  model robot akan diujikan dengan model pengguna serta model-model objek lain secara virtual.
Agar pengujian yang dilakukan di simulasi bisa diterapkan pada \emph{real robot},
  sistem kontroler yang ada pada robot akan dibuat secara terabstraksi dengan memisah setiap komponen menjadi \emph{nodes} menggunakan ROS 2.
Hasilnya,
  sistem yang dibuat mampu menghasilkan tindakan yang sama dalam menggerakkan model robot dan \emph{real robot} dengan perbedaan \emph{error} sebesar 2.6\% di simulasi dan 12.5\% di dunia nyata.
Dengan performa yang dimiliki ROS 2,
  pengiriman citra dengan resolusi hingga 640 x 480 mampu menghasilkan \emph{delay} di bawah 50 ms dan frekuensi di atas 90\% pada sesama perangkat maupun antar-perangkat.
Pada simulasi,
  model pengguna terbukti mampu digunakan untuk mensimulasikan pengguna melalui hasil percobaan deteksi pose,
  sedangkan lingkungan simulasi terbukti mampu digunakan untuk mensimulasikan ruangan melalui hasil percobaan SLAM.

Kata Kunci: Simulasi, \emph{Assistive Robotics}, ROS2, Gazebo.
