\begin{center}
  \large\textbf{ABSTRAK}
\end{center}

\vspace{2ex}

\begingroup
  \setlength{\tabcolsep}{0pt}
  \noindent
  \begin{tabularx}{\textwidth}{l >{\centering}m{2em} X}
    Nama        &:& Muhammad Alfi Maulana Fikri \\
    Judul       &:&	Pengembangan Lingkungan Simulasi untuk Pengujian \emph{Socially Assistive Robots} Menggunakan ROS 2 dan Gazebo \\
    Pembimbing  &:& 1. Prof. Dr. Ir. Mauridhi Hery Purnomo, M.Eng. \\
                & & 2. Dr. I Ketut Eddy Purnama, S.T., M.T. \\
  \end{tabularx}
\endgroup

Selama beberapa tahun terakhir,
  robot telah mengalami perkembangan yang cukup signifikan.
Salah satu bentuk perkembangan tersebut adalah \emph{socially assistive robots} (SARs) yang mampu memberikan bantuan kepada pengguna dalam bentuk interaksi sosial.
Namun, karena sifatnya yang melibatkan interaksi langsung dengan pengguna,
  pengujian pada SARs akan menjadi sulit dan beresiko.
Untuk itu, pada penelitian ini kami mengajukan lingkungan simulasi untuk pengujian SARs yang dibuat menggunakan simulator Gazebo.
% Di dalam lingkungan simulasi ini,
%   model robot akan diujikan dengan model pengguna serta model-model objek lain secara virtual.
Agar pengujian yang dilakukan di simulasi bisa diterapkan pada robot fisik,
  sistem kontroler yang ada pada robot akan dibuat secara terabstraksi dengan memisah setiap komponen menjadi \emph{nodes} menggunakan ROS 2.
% Dengan adanya abstraksi tersebut,
%   program utama robot dapat digunakan pada berbagai sistem yang ada,
%   terlepas dari sistem itu ada di simulasi maupun ada pada robot fisik.
Hasilnya,
  lingkungan simulasi yang dibuat dapat digunakan untuk melakukan pengujian SARs secara virtual.
selain itu, ketika diujikan pada robot fisik,
  tindakan yang dihasilkan memiliki kesamaan dengan yang dihasilkan oleh model robot di simulasi.

Kata Kunci: Simulasi, \emph{Assistive Robotics}, ROS2, Gazebo.
