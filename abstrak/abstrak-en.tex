\begin{center}
  \large\textbf{ABSTRACT}
\end{center}

\addcontentsline{toc}{chapter}{ABSTRACT}
\vspace{2ex}

\begingroup
  \setlength{\tabcolsep}{0pt}
  \noindent
  \begin{tabularx}{\textwidth}{l >{\centering}m{2em} X}
    \emph{Name}     &:& Muhammad Alfi Maulana Fikri \\
    \emph{Title}    &:&	\emph{Development of Simulation Environment for Socially Assistive Robots Testing Using ROS 2 and Gazebo} \\
    \emph{Advisors} &:& 1. Prof. Dr. Ir. Mauridhi Hery Purnomo, M.Eng. \\
                    & & 2. Dr. I Ketut Eddy Purnama, S.T., M.T. \\
  \end{tabularx}
\endgroup

\emph{
  In this study, we propose a simulation environment developed using ROS 2 and Gazebo for socially assistive robots (SARs) testing.
  In this simulation environment, the robot model used will be tested with the user model and other virtual object models.
  To make it easier to transfer the program from simulation to physical robot, the robot controller will be developed separately from the simulation environment in which when testing, both will be connected to each other using ROS 2 interprocess communication system.
  It is expected that the simulation environment created can assist the testing of SARs by minimizing risk, reducing costs, and saving time when compared to conducting direct testing using physical robots.
}

\emph{Keywords}: \emph{Simulation}, \emph{Assistive Robotics}, ROS 2, Gazebo.
