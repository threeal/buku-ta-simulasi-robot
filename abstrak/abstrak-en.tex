\begin{center}
  \large\textbf{ABSTRACT}
\end{center}

\vspace{2ex}

\begingroup
  \setlength{\tabcolsep}{0pt}
  \noindent
  \begin{tabularx}{\textwidth}{l >{\centering}m{2em} X}
    \emph{Name}     &:& Muhammad Alfi Maulana Fikri \\
    \emph{Title}    &:&	\emph{Development of Simulation Environment for Socially Assistive Robots Testing Using ROS 2 and Gazebo} \\
    \emph{Advisors} &:& 1. Prof. Dr. Ir. Mauridhi Hery Purnomo, M.Eng. \\
                    & & 2. Dr. I Ketut Eddy Purnama, S.T., M.T. \\
  \end{tabularx}
\endgroup

\emph{
  Over the past few years,
    robots have undergone significant developments.
  One of this development is socially assistive robots (SARs) which are able to assist users in the form of social interaction.
  However, due to their nature which involves direct interaction with the user,
    testing SARs could be difficult and risky.
  For this reason, in this study we propose a simulation environment for testing SARs created using the Gazebo simulator.
  In order for the test performed in the simulation could be applied to real robots,
    the controller system in the robot will be abstracted by separating each component into nodes using ROS 2.
  As a result, the created simulation environment could be used to test SARs virtually.
    In addition, when tested on a real robot,
    the resulting actions are similar to those generated by the robot model in the simulation.
}

\emph{Keywords}: \emph{Simulation}, \emph{Assistive Robotics}, ROS 2, Gazebo.
