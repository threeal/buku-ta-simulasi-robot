\begin{center}
  \large\textbf{ABSTRACT}
\end{center}

\vspace{2ex}

\begingroup
  \setlength{\tabcolsep}{0pt}
  \noindent
  \begin{tabularx}{\textwidth}{l >{\centering}m{2em} X}
    \emph{Name}     &:& Muhammad Alfi Maulana Fikri \\
    \emph{Title}    &:&	\emph{Development of Simulation Environment for Socially Assistive Robots Testing Using ROS 2 and Gazebo} \\
    \emph{Advisors} &:& 1. Prof. Dr. Ir. Mauridhi Hery Purnomo, M.Eng. \\
                    & & 2. Dr. I Ketut Eddy Purnama, S.T., M.T. \\
  \end{tabularx}
\endgroup

\emph{
  Over the past few years,
    robots have undergone significant developments.
  One of this development is socially assistive robots (SARs) which are able to assist users in the form of social interaction.
  However,
    due to their nature which involves direct interaction with the user,
    testing SARs could be difficult and risky.
  For this reason,
    in this study we propose a simulation environment for testing SARs created using the Gazebo simulator.
  In this simulation environment,
    the robot model will be tested virtually with a user model and other object models.
  In order for the test performed in the simulation could be applied to real robots,
    the controller system in the robot will be abstracted by separating each component into nodes using ROS 2.
  As a result,
    the system created can produce the same action in moving the robot model and the real robot with an error difference of 2.6\% in the simulation and 12.5\% in the real world.
  With the performance of ROS 2,
    images delivery with a resolution of up to 640 x 480 can produce delays below 50 ms and frequencies above 90\% on the same device and between devices.
  In the simulation,
    the user model proved capable of being used to simulate the user through the results of the pose detection experiment,
    while the simulation environment proved capable of being used to simulate the room through the results of the SLAM experiment.
}

\emph{Keywords}: \emph{Simulation}, \emph{Assistive Robotics}, ROS 2, Gazebo.
