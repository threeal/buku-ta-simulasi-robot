\begin{center}
  \Large
  \textbf{KATA PENGANTAR}
\end{center}

\vspace{2ex}

Puji dan syukur kehadirat Allah SWT atas segala limpahan berkah, rahmat, serta hidayah-Nya, penulis  dapat menyelesaikan penelitian ini dengan judul
``\textbf{Pengembangan Lingkungan Simulasi untuk Pengujian \emph{Socially Assistive Robots} Menggunakan ROS 2 dan Gazebo}''.
Penelitian ini disusun dalam rangka pemenuhan bidang riset di Departemen Teknik Komputer,
  serta digunakan sebagai persyaratan menyelesaikan pendidikan S1.

Dalam penyusunan buku ini,
  penulis mengucapkan terima kasih kepada Keluarga yang telah memberikan dorongan spiritual dan material dalam penyelesaian penelitian ini.
Terutama kepada Ayah atas didikannya kepada penulis selama ini,
  semoga beliau husnul khatimah di sana, aamiin.

Penulis juga mengucapkan terima kasih kepada Bapak Prof. Dr. Ir. Mauridhi Hery Purnomo, M.Eng.,
  Bapak Dr. I Ketut Eddy Purnama, S.T., M.T.,
  dan Bapak Muhtadin ST., MT. atas arahan dan bimbingan selama pengerjaan penelitian tugas akhir ini.
Serta kepada Bapak-ibu dosen pengajar Departemen Teknik Komputer atas pengajaran dan perhatian yang diberikan kepada penulis selama ini.

Dan terakhir,
  terima kasih kepada rekan-rekan ICHIRO ITS, Robotika ITS, dan B201 crew atas pengalamannya kepada penulis.
Serta kepada rekan-rekan seperjuangan Teknik Komputer 2017, E57, dan penghuni rumah anak TK.

Kesempurnaan hanya milik Allah SWT, untuk itu penulis memohon segenap kritik dan saran yang  membangun.
Semoga penelitian ini dapat memberikan manfaat bagi kita semua, aamiin.

\vspace{4ex}

\begin{flushright}
  \begin{tabular}[b]{c}
    Surabaya, Agustus 2021\\
    \\
    \\
    \\
    \\
    Muhammad Alfi Maulana Fikri
  \end{tabular}
\end{flushright}
