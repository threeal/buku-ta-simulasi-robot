\begin{center}
  \Large
  \textbf{KATA PENGANTAR}
\end{center}

\vspace{2ex}

Puji dan syukur kehadirat Allah SWT atas segala limpahan berkah, rahmat, serta hidayah-Nya, penulis  dapat menyelesaikan penelitian ini dengan judul
``\textbf{Pengembangan Lingkungan Simulasi untuk Pengujian \emph{Socially Assistive Robots} Menggunakan ROS 2 dan Gazebo}''.

Penelitian ini disusun dalam rangka pemenuhan bidang riset di Departemen Teknik Komputer, serta digunakan sebagai persyaratan menyelesaikan pendidikan  S1.
Penelitian ini dapat terselesaikan tidak lepas dari bantuan berbagai pihak.
Oleh karena itu, penulis mengucapkan terima kasih kepada:

\begin{enumerate}[nolistsep]

  \item Keluarga, Bapak, Ibu dan Saudara tercinta yang telah memberikan dorongan spiritual dan material dalam penyelesaian penelitian ini.

  \item Bapak Dr. Supeno Mardi Susiki Nugroho, ST., MT. selaku Kepala Departemen Teknik Komputer - FTEIC ITS.

  \item Bapak Prof. Dr. Ir. Mauridhi Hery Purnomo, M.Eng., Bapak Dr. I Ketut Eddy Purnama, S.T., M.T., dan Bapak Muhtadin ST., MT. atas arahan dan bimbingan selama pengerjaan penelitian tugas akhir ini.

  \item Bapak-ibu dosen pengajar Departemen Teknik Komputer, atas pengajaran,  bimbingan, serta perhatian yang diberikan kepada penulis selama ini.

  \item Seluruh rekan-rekan ICHIRO-ITS, B201 crew, dan penghuni rumah anak TK.

\end{enumerate}

Kesempurnaan hanya milik Allah SWT, untuk itu penulis memohon segenap kritik dan saran yang  membangun.
Semoga penelitian ini dapat memberikan manfaat bagi kita semua.
Amin.

\begin{flushright}
  \begin{tabular}[b]{c}
    Surabaya, Juni 2021\\
    \\
    \\
    \\
    \\
    Muhammad Alfi Maulana Fikri
  \end{tabular}
\end{flushright}
