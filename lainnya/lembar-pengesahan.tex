\begin{center}
	\large
  \textbf{LEMBAR PENGESAHAN}
\end{center}

\thispagestyle{empty}

\begin{center}
  \textbf{PENGEMBANGAN LINGKUNGAN SIMULASI UNTUK PENGUJIAN \emph{SOCIALLY ASSISTIVE ROBOTS} MENGGUNAKAN ROS 2 DAN GAZEBO}
\end{center}

\begingroup
  \small

  \begin{center}
    Tugas Akhir ini disusun untuk memenuhi salah satu syarat memperoleh gelar Sarjana Teknik di Institut Teknologi Sepuluh Nopember Surabaya
  \end{center}

  \begin{center}
    Oleh: Muhammad Alfi Maulana Fikri (NRP. 0721 17 4000 0009)
  \end{center}

  \begingroup
    \setlength{\tabcolsep}{0pt}
    \noindent
    \begin{tabularx}{\textwidth}{X r}
    Tanggal Ujian : 12 Juni 2021 & Periode Wisuda : September 2021
    \end{tabularx}
  \endgroup

  \begin{center}
    Disetujui Oleh:
  \end{center}

  \begingroup
    \setlength{\tabcolsep}{0pt}
    \noindent
    \begin{tabularx}{\textwidth}{X c}
      Prof. Dr. Ir. Mauridhi Hery Purnomo, M.Eng. & (Pembimbing I) \\
      NIP: 9580916 198601 1 001                   & \multido{}{35}{.} \\
      & \\
      & \\
      Dr. I Ketut Eddy Purnama, S.T., M.T.        & (Pembimbing II) \\
      NIP: 9690730 199512 1 001                   & \multido{}{35}{.} \\
      & \\
      & \\
      \multido{}{70}{.}                           & (Penguji I) \\
      NIP: \multido{}{61}{.}                      & \multido{}{35}{.} \\
      & \\
      & \\
      \multido{}{70}{.}                           & (Penguji II) \\
      NIP: \multido{}{61}{.}                      & \multido{}{35}{.} \\
      & \\
      & \\
      \multido{}{70}{.}                           & (Penguji III) \\
      NIP: \multido{}{61}{.}                      & \multido{}{35}{.} \\
    \end{tabularx}
  \endgroup

  \vspace{2ex}

  \begin{center}
    Mengetahui, \\
    Kepala Departemen Teknik Komputer \\

    \vspace{8ex}

    \underline{Dr. Supeno Mardi Susiki Nugroho, S.T., M.T.} \\
    NIP. 9700313 199512 1 001
  \end{center}
\endgroup
